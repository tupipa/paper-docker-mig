
In this section, we dive into the inner details of Docker container's layered file system and show our design of the  service hand-off framework based on live migration of the containers.  Generally, our architecture supports high speed offloading service hand-off through the following stages:
% \begin{enumerate*}[series = tobecont, itemjoin = \quad]

\begin{enumerate}[label=\textbf{S\arabic*}]

\item Dynamically predict the possible target servers and pre-download the offloading application's Docker images on each target servers.

\item \label{predump} Checkpoint the container's memory and keep the container running. At the same time, send checkpointed memory images to the target servers. 

\item \label{prepare} Upon receiving the hand-off request to a certain target server, prepare for file system synchronization by remapping the image layers' local identifications on the target server.

% \item \label{firstsync} Upon receiving migration request, compress the thin container layer and send to target.

\item \label{checkpoint} Checkpoint and stop the container. From this point the container will stop running on the source server and waiting to be restored on the target server.

\item \label{img-sync} Get the different of checkpointed memory images between \ref{predump} and \ref{checkpoint}, and send the difference to the target server. 

\item \label{fs-sync} Compress and send the thin container layer to the target host.
% \item \label{finalsync} Synchronize the file system for the last time, updating only the changed files of the container layer before last synchronization in \ref{firstsync}.

\item \label{restore} Restore container on the target host.

\end{enumerate}

Before we start to discuss our design, we need to first introduce some inner details about Docker and how it manages it's layered storage. Although Docker is becoming more and more popular and widely adopted in the industrial world, the inner details about how it manages the layered storage is still not well-documented. As far as we know, This is the first paper that illustrates the inner details of Docker layered storage and leverage it to speed up the live migration of Docker containers.





\subsection{Docker Layered Storage Management}\label{aufsIntroduction}

As we mentioned above, Docker use layered storage for its containers. Each Docker image references a list of read-only layers that represent file system differences. Layers are stacked on top of each other to form a base for a container’s root filesystem \cite{dockerlayer}. 
% \cite{https://docs.docker.com/engine/userguide/storagedriver/imagesandcontainers/#images-and-layers}

% ~\\
% \bigskip
% \smallskip
\smallbreak
\subsubsection{\textit{Container Layer} and Base Image Layers}
~\smallbreak
When a new container is created, a new, thin, writable storage layer is created on top of the underlying image layer stack, which is often called the \textit{``container layer''}. All changes made to the running container -- such as writing new files, modifying existing files, and deleting files - are written to this thin writable \textit{container layer}\cite{dockerlayer}.
% \def\cnt{16}
% \def\cnt{3}
\def\boxh{0.5}
\def\boxw{6.5}
\begin{figure}
    \centering
    
\tikzset{
    block/.style = {rectangle, draw, text width=6.0cm, line width=0.75pt,minimum height=0.6cm},
    block_dash/.style = {block, dashed},
}
\begin{tikzpicture}[node distance=0cm, auto]
\node (st16) [block_dash,
anchor=north east, 
% anchor=south, 
label={[label distance=0.3cm]-180:R/W}]
{febfb1642ebeb25857bf2a9c558bf695-init};
\node (st15) [block ,below =0.1cm of st16,label={[label distance=0.3cm]-180:RO}] 
{fac86d61dfe33f821e8d0e7660473381};
\node (st14) [block ,below = of st15,label={[label distance=0.3cm]-180:RO}] 
{984034c1bb9c62ac63fff949a70d1c06};
\node (st13) [block ,below = of st14,label={[label distance=0.3cm]-180:RO}] 
{2de00a5b0fb59d8eb7301b7523d96d3e};
\node (st12) [block ,below = of st13,label={[label distance=0.3cm]-180:RO}] {0cff6d24b7f45835d42401ec28408b34};
\node (st11) [block ,below = of st12,label={[label distance=0.3cm]-180:RO}] {edc8651e3133e0c7ad4b143e7aa8a238};
\node (st10) [block ,below = of st11,label={[label distance=0.3cm]-180:RO}] {d3eb4d3be5e23cad808eaa419d1f5302};
\node (st9) [block ,below = of st10,label={[label distance=0.3cm]-180:RO}] {374b33d90d684194b17796430fdf8dee};
\node (st8) [block ,below = of st9,label={[label distance=0.3cm]-180:RO}] {76bbec466a0b0b306ab1374476dc11f4};
\node (st7) [block ,below = of st8,label={[label distance=0.3cm]-180:RO}] {1193d29ec78769d161661434acb6d4ed};
\node (st6) [block ,below = of st7,label={[label distance=0.3cm]-180:RO}] {788c79301df4cbac57ffb834d6abded8};
\node (st5) [block ,below = of st6,label={[label distance=0.3cm]-180:RO}] {1562b357dde64e8ed8202c61959d6b13};
\node (st4) [block ,below = of st5,label={[label distance=0.3cm]-180:RO}] {44e0c44aca426478341b9c9f87f05f34};
\node (st3) [block ,below = of st4,label={[label distance=0.3cm]-180:RO}] {27a0a4ac3045470182cf7c1f4c26e379};
\node (st2) [block,below = of st3,label={[label distance=0.3cm]-180:RO}] {87b1dd26596e8e78e294a47b6b3fc3e9};
\node (st1) [block ,below = of st2, label={ [label distance=0.3cm]-180:RO}] {80db20d8e37dc3795b17e0e59930a408
};
\end{tikzpicture}
    \caption{OpenFace Container's Image Layer Stack. Container's  ID is \textit{
    9ec79a095ef4db1fc5edc5e4059f5a10}. It's rootfs ID is stored it's meta data file \textit{state.json}. The stack list is stored in file \textit{./aufs/layers/febfb1642ebeb25857bf2a9c558bf695}. On the top is the container layer, the only writable storage layer fot this container.}
    \label{fig:stacklist}
\end{figure}













% \begin{figure}
%     \centering
% \begin{tikzpicture}
% \draw [dashed] (0,15*\boxh + 0.1) rectangle (\boxw,16*\boxh + 0.2) node[pos=.5,label={[label distance=0.3cm]-180:R/W}]
% {febfb1642ebeb25857bf2a9c558bf695-init};
% \draw (0,14*\boxh) rectangle (\boxw,15*\boxh) node[pos=.5,label={[label distance=0.3cm]-180:RO}] 
% {fac86d61dfe33f821e8d0e7660473381};
% \draw (0,13*\boxh) rectangle (\boxw,14*\boxh) node[pos=.5,label={[label distance=0.3cm]-180:RO}] 
% {984034c1bb9c62ac63fff949a70d1c06};
% \draw (0,12*\boxh) rectangle (\boxw,13*\boxh) node[pos=.5,label={[label distance=0.3cm]-180:RO}] 
% {2de00a5b0fb59d8eb7301b7523d96d3e};
% \draw (0,11*\boxh) rectangle (\boxw,12*\boxh) node[pos=.5,label={[label distance=0.3cm]-180:RO}] {0cff6d24b7f45835d42401ec28408b34};
% \draw (0,10*\boxh) rectangle (\boxw,11*\boxh) node[pos=.5,label={[label distance=0.3cm]-180:RO}] {edc8651e3133e0c7ad4b143e7aa8a238};
% \draw (0,9*\boxh) rectangle (\boxw,10*\boxh) node[pos=.5,label={[label distance=0.3cm]-180:RO}] {d3eb4d3be5e23cad808eaa419d1f5302};
% \draw (0,8*\boxh) rectangle (\boxw,9*\boxh) node[pos=.5,label={[label distance=0.3cm]-180:RO}] {374b33d90d684194b17796430fdf8dee};
% \draw (0,7*\boxh) rectangle (\boxw,8*\boxh) node[pos=.5,label={[label distance=0.3cm]-180:RO}] {76bbec466a0b0b306ab1374476dc11f4};
% \draw (0,6*\boxh) rectangle (\boxw,7*\boxh) node[pos=.5,label={[label distance=0.3cm]-180:RO}] {1193d29ec78769d161661434acb6d4ed};
% \draw (0,5*\boxh) rectangle (\boxw,6*\boxh) node[pos=.5,label={[label distance=0.3cm]-180:RO}] {788c79301df4cbac57ffb834d6abded8};
% \draw (0,4*\boxh) rectangle (\boxw,5*\boxh) node[pos=.5,label={[label distance=0.3cm]-180:RO}] {1562b357dde64e8ed8202c61959d6b13};
% \draw (0,3*\boxh) rectangle (\boxw,4*\boxh) node[pos=.5,label={[label distance=0.3cm]-180:RO}] {44e0c44aca426478341b9c9f87f05f34};
% \draw (0,2*\boxh) rectangle (\boxw,3*\boxh) node[pos=.5,label={[label distance=0.3cm]-180:RO}] {27a0a4ac3045470182cf7c1f4c26e379};
% \draw (0,\boxh) rectangle (\boxw,2*\boxh) node[pos=.5,label={[label distance=0.3cm]-180:RO}] {87b1dd26596e8e78e294a47b6b3fc3e9};
% \draw (0,0) rectangle (\boxw,\boxh) node[ pos=0.5,
% % anchor=east, 
% label={[label distance=0.3cm]-180:RO}] {80db20d8e37dc3795b17e0e59930a408
% };
% \end{tikzpicture}
%     \caption{Image Layer Stack Example}
%     \label{fig:stacklist}
% \end{figure}

For example, 
Figure~\ref{fig:stacklist} shows the stacked image layers for OpenFace application. The dashed box on the top is \textit{container layer} for openface. Underlying are base image layers. To resolve the access request for a file name, the storage driver will search the file name from top layer towards the bottom layer, the first copy of the file will be returned for accessing, regardless the same file name on the underlying layers.

% It greatly simplifies the process of setting environment for any software platform. 

% \subsection{Docker Container File System Structure and Synchronization based on AUFS storage driver} 


\smallbreak
\subsubsection{Image Layer ID Local Mapping}
~\smallbreak
Since Docker 1.10, all the image and layer data are addressed by secure content SHA256 hash ID. 
This addressable image design is supposed to improve security by avoiding ID collisions, and maintaining data integrity after pull, push, load, and save operations. It enables better sharing of layers by allowing many images to freely share their layers locally even if they didn’t come from the same build\cite{dockerlayer}. 

However, due to the active development of Docker platform and its relatively young ages of history, there is no available articles illustrating how those addressable images worked by those SHA256 hash. 
% original and cache IDs are used and maintained throughout the life cycle of one container, as well as the life cycle of the underlying docker daemon. 

By investigating into the source code of Docker as well as its storage drivers, we find out how Docker platform addresses different storage layers and propose our methods to leverage those layers in the live migration of Docker containers.
% keeps track of those mapping relationship. 

For example, we find that 
if the same image is downloaded from the same build on the cloud, Docker will mapping the layers' original layer IDs to a new secure content hash, called `cache ID'. Every image layer's original ID will be replaced with its cache ID. From then on, Docker daemon will address the image layer by this cache ID when it creates, starts, stops, or checkpoints/restores a container. 


\smallbreak
\subsubsection{ID Matching Between Docker Host} \label{intro:idMatching}
~\smallbreak

As we discussed above, 
the image layer on the cloud will have different cache IDs when downloaded to different Docker host. 
% It becomes a challenge to resolve those addressable IDs when we migrate a container between different hosts. 
Therefore, if we want to share the common image layers between different Docker hosts, we need to resolve the image layers mapping on different Docker host. 

In order to do this, on each Docker host we map back its cache IDs to their original layer IDs by querying the Docker image database. 
% that have the same origin on two different Docker host.
Then we could match the image layers by its original IDs instead of the cache IDs exposed to the local Docker daemons. More detail could be found in section \ref{idremapping}.

% In order to do this, we then need to generate a new image layer stack with different cache IDs but actually have the same content inside since they have the same original layer IDs. 


% In the Docker's root system, the /usr/lib/docker/ by default (or /usr/lib/docker/0.0/ for docker 1.10 version), 

\smallbreak
\subsubsection{Docker's Storage Driver Backend}
~\smallbreak

Docker delegate the task of managing the container's layered file system to its pluggable storage driver. 
The Docker storage driver is designed to provide a single unified view from a stack of image layers.
Users can choose different kinds of storage driver that are best for their environment and particular use-cases. 

In order to understand some inner details of how those addressable images work, we investigate the source code of Docker system along with one of its most popular storage driver, AUFS. We will show our findings inside the structure of the Docker's storage systems in this paper. For other storage drivers like Btrfs, Device Mapper, overlay, and ZFS, they implement the management of image layers and the \textit{container layer} in their own unique ways, but our framework could also be extended to those drivers. Due to limited time and space, we only conduct experiments on AUFS. The following section will discuss the inner details about our findings inside the Docker's AUFS storage driver.

% TODO: more details in images management, including the mapping details. Draw a docker root directory structure Figure

% \paragraph{AUFS storage}\mbox{}\\

\subsection{AUFS Storage: A Case Study}
The default storage driver on Docker is AUFS. Therefore, we take this as an example to introduce the layered images management.

AUFS storage driver implements Docker image layers by a union mount system. Union Mount is a way of combining numerous directories into one directory that looks like it contains the content from all the them \cite{aufs}. Using Union filesystems merge all the files for each image layer together and presents them as one single read-only directory at the union mount point. If there are duplicate files in different layers, only the file on the higher level layer is available.

AUFS driver has three directories: layers, diff, and mnt. `layers' directory stores the metadata of how image layers are stacked together. `diff' directory stores the exact data inside each layers. `mnt' directory stores the mount points for the root file system of all the containers. 

%  https://github.com/docker/docker/blob/b248de7e332b6e67b08a8981f68060e6ae629ccf/daemon/graphdriver/aufs/aufs.go
% /*
% aufs driver directory structure
%   .
%   ├── layers // Metadata of layers
%   │   ├── 1
%   │   ├── 2
%   │   └── 3
%   ├── diff  // Content of the layer
%   │   ├── 1  // Contains layers that need to be mounted for the id
%   │   ├── 2
%   │   └── 3
%   └── mnt    // Mount points for the rw layers to be mounted
%       ├── 1
%       ├── 2
%       └── 3
% */

 \tikzset{/forest,
    % symlink/.append style={
    %   opacity=.25,
    %   text opacity=.5,
    %   before drawing tree={
    %     {tikz+={\draw [thick, -{>[]}] (!#1.west) ++(4pt,-1.5pt) arc (315:120:5pt);}}
    %   },
    % },
    % file/.append style={
    file/.style={
        % minimum width=0.5cm,
        % minimum height=0.5cm,
        % parent anchor=south,
        % child anchor=north,
        % sibling distance = 4cm,
        % dogeared,
        % fill=white!50,
        % draw,
        trapezium,
        trapezium angle=0,
        rounded corners=2pt,
        draw,
        fill=white!50,
    },
   }
  
  
\begin{figure*}
\centering
% http://tex.stackexchange.com/questions/5073/making-a-simple-directory-tree
\begin{forest}
for tree={
    minimum width=0.5cm,
    minimum height=0.5cm,
    parent anchor=south,
    child anchor=north,
    trapezium,
    trapezium angle=0,
    rounded corners=2pt,
    draw,
    fill=blue!50,
    % % http://tex.stackexchange.com/questions/278708/center-root-of-forest-tree
    % if level=0{ 
    % parent anchor=south,
    % child anchor=north,
    % align=center,
    % l=1cm,
    % fill=white,
    % minimum width=\linewidth,
    % inner xsep=0pt,
    % outer xsep=0pt,
    % }{},
    % myfolder,
}
% [, phantom, s sep = 1cm
[/var/lib/docker/0.0/
[containers
    [<conID>
        [config.v2.json, file]
        % [resolve.conf, file]
        [..., file]
    ]
    % [<conID1>
    %     % [config.v2.json, file]
    %     % % [hostname, file]
    %     % [..., file]
    % ]
]   
[aufs/
    [mnt/
        [<rootfs ID>/
            % [/boot/
            %     [...]
            % ]
            [/etc/
                [hostname, file]
                [hosts, file]
            ]
            [/home/
                % [username/
                %     [.bashrc, file]
                %  [...]
                % ]
            ]
            % [/.../]
        ]
    ]
    [layers/
        [<rootfs ID>, file]
    ]
    [diff/
        [<layer ID1>/
            [/etc/
                [hostname, file]
            ]
            % [/.../]
        ]
        [<layerID2>/
            [/etc/
                [hostname, file]
                [hosts, file]
            ]
        ]
    ]
]
[image/aufs/layerdb/sha256
    % [aufs
        % [layerdb
            % [mounts
            %     % [<conID>]
            % ]
            % [sha256
                [<O-layerID>
                    [cache-id, file]
                    % [diff,file]
                    [parent,file]
                    [..., file]
                ]
            % ]
        % ]
        % [imagedb]
    % ]
]
]
\end{forest}

\caption{Docker Layered File System Structure Based on AUFS Storage Driver}
\label{fig:aufs}
\end{figure*}


%  https://github.com/docker/docker/blob/b248de7e332b6e67b08a8981f68060e6ae629ccf/daemon/graphdriver/aufs/aufs.go
% /*
% aufs driver directory structure
%   .
%   ├── layers // Metadata of layers
%   │   ├── 1
%   │   ├── 2
%   │   └── 3
%   ├── diff  // Content of the layer
%   │   ├── 1  // Contains layers that need to be mounted for the id
%   │   ├── 2
%   │   └── 3
%   └── mnt    // Mount points for the rw layers to be mounted
%       ├── 1
%       ├── 2
%       └── 3
% */

\begin{figure}
\centering
% http://tex.stackexchange.com/questions/5073/making-a-simple-directory-tree
\begin{forest}
for tree={
    minimum width=0.5cm,
    minimum height=0.5cm,
    parent anchor=south,
    child anchor=north,
    trapezium,
    trapezium angle=0,
    rounded corners=2pt,
    draw,
    fill=blue!50,
    % % http://tex.stackexchange.com/questions/278708/center-root-of-forest-tree
    % if level=0{ 
    % parent anchor=south,
    % child anchor=north,
    % align=center,
    % l=1cm,
    % fill=white,
    % minimum width=\linewidth,
    % inner xsep=0pt,
    % outer xsep=0pt,
    % }{},
    % myfolder,
}
% [, phantom, s sep = 1cm
[/var/run/docker/execdriver/native/
    [<conID1>
        [state.json, file]
    ]
    [<conID2>
        [state.json, file]
    ]
]
% ]
\end{forest}

\caption[Caption for LOF]{Runtime Data for Containers\protect\footnotemark 
% -dev, the latest Docker (version 17.04.0-ce, build 4845c56) has this directory changed to \textit{/var/run/docker/libcontainerd/containerd}}
% \footnote{This is for Docker 1.10\-dev, the latest Docker (version 17.04.0\-ce, build 4845c56) has this directory changed to \textit{/var/run/docker/libcontainerd/containerd}}
}
\label{fig:aufs-runtime}
\end{figure}

% \footnotetext{This is for Docker 1.10-dev, the latest Docker (version 17.04.0-ce, build 4845c56) has this directory changed to \textit{/var/run/docker/libcontainerd/containerd}}

Figure~\ref{fig:aufs} shows the Docker storage structure based on AUFS driver. White box stands for a file and blue box stands for a directory. Since all directories share the same parent dir \textit{/var/lib/docker/0.0}, we will use `.' to represent this common directory in the following sections. 


\smallbreak
\subsubsection{Container's Image Layer Stack List}
~\smallbreak
% \smallbreak  \subsubsection{Container's Image Layer Stack List:}

We know that each Docker image contains several image layers. Those image layers are addressed by their SHA256 content hash IDs. Each Docker image has a list that stores all these layer IDs in the order of how they stacked from top to bottom.
The file \textit{./aufs/layers/<rootfs ID>-init}  in Figure \ref{fig:aufs} shows the path of this file inside the AUFS storage structure. This file stores a list of SHA256 IDs of all image layers that will be bind mounted together as the container's root file system. For example, for one container OpenFace with rootfs ID of 
\textit{
febfb1642ebeb25857bf2a9c558bf695
\footnotetext{This is for Docker 1.10-dev, the latest Docker (version 17.04.0-ce, build 4845c56) has the runtime data directory changed to \textit{/var/run/docker/libcontainerd/containerd/}}
\footnote{SHA256 ID has 64 hexadecimal characters, here we truncate it to 32 hexadecimal characters in order to save space}
}, it's stack list file\textit{./aufs/ layers/ febfb1642ebeb25857bf2a9c558bf695}

There is another file \textit{./aufs/layers/<rootfs ID>} is the file stores this list of image layer IDs for that Docker image as well as the ID of the writable \textit{container layer}.  The \textit{container layer}'s ID is used to address the thin writable layer for that container. This \textit{container layer} ID is the same as the <rootfs ID>. 
When the Docker daemon start or restore a container, it will refer to those two files to get a list of all underlying Docker image layer IDs and the \textit{container layer} ID. Then it will resolve those addressable IDs and union mount all those layer stacks together with its thin writable \textit{container layer}. After this, the container will get the union mount of its root file system view under its root mount point. 
Now we see this file behaves like an important handler for all the union file systems for the container. If this file is missing, one container will not be able to union mount the layered file system.

% --------------

\smallbreak 
\subsubsection{Image Layer Content Directory}
~\smallbreak
Now we have the mount point, and a list of image layer IDs as the addresses for the storage layers. But before a container can start, the Docker daemon needs to know where to find the files inside each image layers. Here comes the directory of
 \textit{./aufs/diff/<layer ID>/}, which stores all the image layer contents from one specific layer identified by a local image \textit{<layer ID>}. If \textit{<layer ID>} is the same as \textit{<rootfs ID>} of one container, then this directory is where the content of \textit{container layer} stores, i.e. the container's thin writable layer. 

\smallbreak 
\subsubsection{Unified Mount Point} 
~\smallbreak
The directory \textit{./aufs/mnt/<rootfs ID>/ }  is the root file system mount point of the container. All the image layers are union mounted to this folder and provide a whole file system view for the container. For example, as shown in Figure \ref{fig:aufs}, when a container is created based on a Linux image, its mount point will contain the root directory contents like \textit{/usr/, /home/, /boot/, etc. }. All those directories are mounted from its underlying layered images. 
Since this directory is a mount point for a running container's file system, it will be only available when the container is running. If the container stops running, all the image layers will be unmounted from this mount point. So it will become an empty directory.

Here, the root file system ID, <rootfs ID>, is also an image layer cache ID of container's thin writable \textit{container layer}. 

\smallbreak  
\subsubsection{Layer ID Mapping}  \label{intro:aufs:layerIDMapping}
~\smallbreak
Until now, the layer IDs we have discussed above are just local SHA256 IDs, or the so called cache IDs, which are generated dynamically when each image layer is downloaded by `docker pull' command. As we discussed above, from then on, Docker daemon will address the image layer use the cache ID instead of its original layer ID.

We find the Docker storage system maintains a mapping relationship between the original layer IDs and its cache IDs. All the cached IDs of image layers are stored in the \textit{/image/aufs/layerdb/sha256} directory.
For example, the file \textit{./image/aufs/layerdb/sha256/<O-layerID>/cache-id} shown in Figure~\ref{fig:aufs} stores the cache ID of the image with original ID <O-layerID>. For example, if a string of \textit{
fac86d61dfe33f821e8d0e7660473381} is stored in a file path \textit{./image/ aufs/layerdb/sha256/6384c447ddd6cd859f9be3b53f8b015c/cache-id}, this means there is an image layer with an original ID of \textit{
6384c447dd-d6cd859f9be3b53f8b015c} and it's cache ID is mapped to \textit{
fac86d61df-e33f821e8d0e7660473381}.


% However, due to the active development of Docker community and its relatively young ages of history, there is no available articles illustrating how those original and cache IDs are used and maintained throughout the life cycle of one container, as well as the life cycle of the underlying docker daemon. By investigating into the source code of Docker as well as its storage system, we find out how Docker platform keeps track of those mapping relationship. 

\smallbreak  
\subsubsection{Container Configuration and Runtime State}
~\smallbreak

Finally, there are directories storing the configuration files and runtime data. Figure~\ref{fig:aufs-runtime} shows the runtime data directory stored for each containers. For one container with ID of \textit{<conID>}, there will be one JSON (JavaScript Object Notation)  file \textit{state.json} that stores the run time state of the container. For example, it stores the init pid of the containers' processes with key ``\textit{init\_process\_pid}'', the root file system mount point path with key ``\textit{rootfs}'', as well as the runtime cgroup and namespace meta data, etc.. 

Along with the runtime data directory, there is another directory inside Docker root directory (\textit{/var/lib/docker/0.0}) that stores the configuration data for each container: \textit{./containers/<conID>/} as shown in Figure~\ref{fig:aufs}. For example, \textit{ config.v2.json} file stores the container's creation time, the command that was run when creating the container, etc..


% TODO: more details in AUFS. Draw an aufs driver directory structure Figure

%   ├── layers // Metadata of layers
%   │   ├── 1
%   │   ├── 2
%   │   └── 3
%   ├── diff  // Content of the layer
%   │   ├── 1  // Contains layers that need to be mounted for the id
%   │   ├── 2
%   │   └── 3
%   └── mnt    // Mount points for the rw layers to be mounted
%       ├── 1
%       ├── 2
%       └── 3


% (Before migration, both the source and target edge server have the application base images downloaded.)
 
\subsection{Remapping Image Layers' Identification}

As we introduced above, in the AUFS-based addressable layers, the local cached ID for each downloaded image layer is stored in 


\subsection{Memory Difference Tracking}
In order to reduce the total memory image size during hand-off, we checkpoint the container and get a snapshot of the container's memory in stage \ref{predump}. The snapshot was transferred before the hand-off starts, just as the application's Docker images, which are also downloaded before the hand-off starts.

Upon the hand-off starts, we checkpoint the container and stop it. Then we do a diff operation on the new memory dump images along with the old snapshot we got from stage \ref{predump}. Then we only need to transfer the difference of the dump memory to the target node. We use Xdelta3 to get the memory difference.

\begin{figure}
\begin{subfigure}[b]{0.45\textwidth}
    % \centering
    \includegraphics[width=\linewidth]
    % {figure/test-tarssh.eps}
    {figure/face-518744dump_iter_11.eps}
    \caption{Memory Dumps}
    \label{fig:dump}
\end{subfigure}

\begin{subfigure}[b]{0.45\textwidth}
    % \centering
    \includegraphics[width=\linewidth]
    % {figure/test-tarssh.eps}
    {figure/face-518744-iter-11.eps}
    \caption{Dirty Memory, original}
    \label{fig:diff-ori}
\end{subfigure}
\begin{subfigure}[b]{0.45\textwidth}
    % \centering
    \includegraphics[width=\linewidth]
    % {figure/test-tarssh.eps}
    {figure/face-518744-iter-11-2.eps}
    \caption{Dirty Memory, adjacent}
    \label{fig:diff-adj}
\end{subfigure}

\begin{subfigure}[b]{0.45\textwidth}
    % \centering
    \includegraphics[width=\linewidth]
    % {figure/test-tarssh.eps}
    {figure/face-518744diffdiff_ori11.eps}
    \caption{Dirty Difference, original}
    \label{fig:diffdiff-adj}
\end{subfigure}
\begin{subfigure}[b]{0.45\textwidth}
    % \centering
    \includegraphics[width=\linewidth]
    % {figure/test-tarssh.eps}
    {figure/face-518744diffdiff_adj11.eps}
    \caption{Dirty Difference, adjacent}
    \label{fig:diffdiff-ori}
\end{subfigure}

\caption{  Dirty Memory Size Analysis for OpenFace. The data shows 11 dumps of the running container of OpenFace, indexed from dump 0 to dump 10. Memory is dumped every 10 seconds. 
Dirty Memory Size of Dump 1~10, Compared to Original Dump 0 .
Dirty Memory Size Between Two Adjacent Dumps. Memory is dumped every 10 seconds.
Size of Difference Between Two Adjacent Dirty Memory from Every three Adjacent Memory Dumps. Memory is dumped every 10 seconds. }
\label{fig:diff-diff}
\end{figure}

% \begin{figure}[tb!]
%     \centering
%     \includegraphics[width=0.9\textwidth]
%     % {figure/test-tarssh.eps}
%     {figure/face-518744-iter-11.eps}
%     \caption{Dirty Memory Size of Dump 1~10, Compared to Original Dump 0 }
%     \label{fig:diff-0}
% \end{figure}

% \begin{figure}[tb!]
%     \centering
%     \includegraphics[width=0.45\textwidth]
%     % {figure/test-tarssh.eps}
%     {figure/face-518744-iter-11-2.eps}
%     \caption{Dirty Memory Size Between Two Adjacent Dumps. Memory is dumped every 10 seconds }
%     \label{fig:diff-adjacent}
% \end{figure}

% \begin{figure}[tb!]
%     \centering
%     \includegraphics[width=0.45\textwidth]
%     % {figure/test-tarssh.eps}
%     {figure/face-518744-diffdiff-11.eps}
%     \caption{Size of Difference Between Two Adjacent Dirty Memory from Every three Adjacent Memory Dumps. Memory is dumped every 10 seconds. }
%     \label{fig:diff-diff}
% \end{figure}


\subsection{Compression of the Transferred Data}

During the live migration of a container, we mainly have 4 kinds of data need to transfer:

The layer stacks information, the container writable layer, the meta data files of the container, and the main memory images checkpointed by CRIU. 

% In order to speed up the synchronization process, we combine both the compression techniques and synchronization tools. 

The layer stacks information is send via RPC socket connection. Those information is only a list of SHA256 ID strings, so it's quite efficient to be sent as a socket message. There is no need to compress it.

The container writable layer and meta data files are regular files in the storage system, so we use bzip2  to compress and send via ssh connection.

The dump memory images are binary data, which is not efficient to compressed using bzip2. So we use ??? to compress and send via ssh connections.

% \ref{predump}, \ref{img-sync}.

% For the root file system and meta data files. We have two stages of files synchronization before and after the container was checkpoints. The first synchronization in \ref{firstsync} will transfer the base file system for the container layer, so it's better to compress it. We use 'tar' command to compress and sent via SSH. For the final synchronization of the file system, since we already have most of the files transferred, we choose to use 'rsync' to send over only the changed files.

% For the checkpointed memory images, we compress them via 'tar' compression and send via SSH connections.

% Syncing smaller files individually through rsync or scp results in each file starting at least one own data packet over the net. If the file is small and the packets are many, this results in increased protocol overhead. Now count in that there are more than one data packets for each file by means of rsync protocol as well (transferring checksums, comparing...), the protocol overhead quickly builds up

% \url{http://unix.stackexchange.com/questions/30953/tar-rsync-untar-any-speed-benefit-over-just-rsync}

TODO: more details about compression and synchronization: more memory images compression techniques (to replace tar cmd). 

TODO: rsync vs. tar comparison

