
In this section, we discuss the background of our work and dive into the inner details of Docker container's layered storage system and show which part of the file system is necessary to be transferred during live migration. 
% Before we start to discuss our design, we need to first introduce some inner details about Docker and how it manages it's layered storage. 
Although Docker is becoming more and more popular and widely adopted in the industrial world, the inner details about how it manages the layered storage is still not well-documented. As far as we know, This is the first paper that illustrates the inner details of Docker layered storage and leverage it to speed up the live migration of Docker containers.




\subsection{Docker Layered Storage Management}\label{aufsIntroduction}

As we mentioned above, Docker use layered storage for its containers. Each Docker image references a list of read-only layers that represent file system differences. Layers are stacked on top of each other to form a base for a container’s root filesystem \cite{dockerlayer}. 
% \cite{https://docs.docker.com/engine/userguide/storagedriver/imagesandcontainers/#images-and-layers}

% ~\\
% \bigskip
% \smallskip
\smallbreak
\subsubsection{\textit{Container Layer} and Base Image Layers}
~\smallbreak
When a new container is created, a new, thin, writable storage layer is created on top of the underlying image layer stack, which is often called the \textit{``container layer''}. All changes made to the running container -- such as writing new files, modifying existing files, and deleting files - are written to this thin writable \textit{container layer}\cite{dockerlayer}.
% \def\cnt{16}
% \def\cnt{3}
\def\boxh{0.5}
\def\boxw{6.5}
\begin{figure}
    \centering
    
\tikzset{
    block/.style = {rectangle, draw, text width=6.0cm, line width=0.75pt,minimum height=0.6cm},
    block_dash/.style = {block, dashed},
}
\begin{tikzpicture}[node distance=0cm, auto]
\node (st16) [block_dash,
anchor=north east, 
% anchor=south, 
label={[label distance=0.3cm]-180:R/W}]
{febfb1642ebeb25857bf2a9c558bf695-init};
\node (st15) [block ,below =0.1cm of st16,label={[label distance=0.3cm]-180:RO}] 
{fac86d61dfe33f821e8d0e7660473381};
\node (st14) [block ,below = of st15,label={[label distance=0.3cm]-180:RO}] 
{984034c1bb9c62ac63fff949a70d1c06};
\node (st13) [block ,below = of st14,label={[label distance=0.3cm]-180:RO}] 
{2de00a5b0fb59d8eb7301b7523d96d3e};
\node (st12) [block ,below = of st13,label={[label distance=0.3cm]-180:RO}] {0cff6d24b7f45835d42401ec28408b34};
\node (st11) [block ,below = of st12,label={[label distance=0.3cm]-180:RO}] {edc8651e3133e0c7ad4b143e7aa8a238};
\node (st10) [block ,below = of st11,label={[label distance=0.3cm]-180:RO}] {d3eb4d3be5e23cad808eaa419d1f5302};
\node (st9) [block ,below = of st10,label={[label distance=0.3cm]-180:RO}] {374b33d90d684194b17796430fdf8dee};
\node (st8) [block ,below = of st9,label={[label distance=0.3cm]-180:RO}] {76bbec466a0b0b306ab1374476dc11f4};
\node (st7) [block ,below = of st8,label={[label distance=0.3cm]-180:RO}] {1193d29ec78769d161661434acb6d4ed};
\node (st6) [block ,below = of st7,label={[label distance=0.3cm]-180:RO}] {788c79301df4cbac57ffb834d6abded8};
\node (st5) [block ,below = of st6,label={[label distance=0.3cm]-180:RO}] {1562b357dde64e8ed8202c61959d6b13};
\node (st4) [block ,below = of st5,label={[label distance=0.3cm]-180:RO}] {44e0c44aca426478341b9c9f87f05f34};
\node (st3) [block ,below = of st4,label={[label distance=0.3cm]-180:RO}] {27a0a4ac3045470182cf7c1f4c26e379};
\node (st2) [block,below = of st3,label={[label distance=0.3cm]-180:RO}] {87b1dd26596e8e78e294a47b6b3fc3e9};
\node (st1) [block ,below = of st2, label={ [label distance=0.3cm]-180:RO}] {80db20d8e37dc3795b17e0e59930a408
};
\end{tikzpicture}
    \caption{OpenFace Container's Image Layer Stack. Container's  ID is \textit{
    9ec79a095ef4db1fc5edc5e4059f5a10}. It's rootfs ID is stored it's meta data file \textit{state.json}. The stack list is stored in file \textit{./aufs/layers/febfb1642ebeb25857bf2a9c558bf695}. On the top is the container layer, the only writable storage layer fot this container.}
    \label{fig:stacklist}
\end{figure}













% \begin{figure}
%     \centering
% \begin{tikzpicture}
% \draw [dashed] (0,15*\boxh + 0.1) rectangle (\boxw,16*\boxh + 0.2) node[pos=.5,label={[label distance=0.3cm]-180:R/W}]
% {febfb1642ebeb25857bf2a9c558bf695-init};
% \draw (0,14*\boxh) rectangle (\boxw,15*\boxh) node[pos=.5,label={[label distance=0.3cm]-180:RO}] 
% {fac86d61dfe33f821e8d0e7660473381};
% \draw (0,13*\boxh) rectangle (\boxw,14*\boxh) node[pos=.5,label={[label distance=0.3cm]-180:RO}] 
% {984034c1bb9c62ac63fff949a70d1c06};
% \draw (0,12*\boxh) rectangle (\boxw,13*\boxh) node[pos=.5,label={[label distance=0.3cm]-180:RO}] 
% {2de00a5b0fb59d8eb7301b7523d96d3e};
% \draw (0,11*\boxh) rectangle (\boxw,12*\boxh) node[pos=.5,label={[label distance=0.3cm]-180:RO}] {0cff6d24b7f45835d42401ec28408b34};
% \draw (0,10*\boxh) rectangle (\boxw,11*\boxh) node[pos=.5,label={[label distance=0.3cm]-180:RO}] {edc8651e3133e0c7ad4b143e7aa8a238};
% \draw (0,9*\boxh) rectangle (\boxw,10*\boxh) node[pos=.5,label={[label distance=0.3cm]-180:RO}] {d3eb4d3be5e23cad808eaa419d1f5302};
% \draw (0,8*\boxh) rectangle (\boxw,9*\boxh) node[pos=.5,label={[label distance=0.3cm]-180:RO}] {374b33d90d684194b17796430fdf8dee};
% \draw (0,7*\boxh) rectangle (\boxw,8*\boxh) node[pos=.5,label={[label distance=0.3cm]-180:RO}] {76bbec466a0b0b306ab1374476dc11f4};
% \draw (0,6*\boxh) rectangle (\boxw,7*\boxh) node[pos=.5,label={[label distance=0.3cm]-180:RO}] {1193d29ec78769d161661434acb6d4ed};
% \draw (0,5*\boxh) rectangle (\boxw,6*\boxh) node[pos=.5,label={[label distance=0.3cm]-180:RO}] {788c79301df4cbac57ffb834d6abded8};
% \draw (0,4*\boxh) rectangle (\boxw,5*\boxh) node[pos=.5,label={[label distance=0.3cm]-180:RO}] {1562b357dde64e8ed8202c61959d6b13};
% \draw (0,3*\boxh) rectangle (\boxw,4*\boxh) node[pos=.5,label={[label distance=0.3cm]-180:RO}] {44e0c44aca426478341b9c9f87f05f34};
% \draw (0,2*\boxh) rectangle (\boxw,3*\boxh) node[pos=.5,label={[label distance=0.3cm]-180:RO}] {27a0a4ac3045470182cf7c1f4c26e379};
% \draw (0,\boxh) rectangle (\boxw,2*\boxh) node[pos=.5,label={[label distance=0.3cm]-180:RO}] {87b1dd26596e8e78e294a47b6b3fc3e9};
% \draw (0,0) rectangle (\boxw,\boxh) node[ pos=0.5,
% % anchor=east, 
% label={[label distance=0.3cm]-180:RO}] {80db20d8e37dc3795b17e0e59930a408
% };
% \end{tikzpicture}
%     \caption{Image Layer Stack Example}
%     \label{fig:stacklist}
% \end{figure}

For example, 
Figure~\ref{fig:stacklist} shows the stacked image layers for OpenFace application. The dashed box on the top is \textit{container layer} for openface. Underlying are base image layers. To resolve the access request for a file name, the storage driver will search the file name from top layer towards the bottom layer, the first copy of the file will be returned for accessing, regardless the same file name on the underlying layers.

% It greatly simplifies the process of setting environment for any software platform. 

% \subsection{Docker Container File System Structure and Synchronization based on AUFS storage driver} 


\smallbreak
\subsubsection{Image Layer ID Local Mapping}
~\smallbreak
Since Docker 1.10, all the image and layer data are addressed by secure content SHA256 hash ID. 
This addressable image design is supposed to improve security by avoiding ID collisions, and maintaining data integrity after pull, push, load, and save operations. It enables better sharing of layers by allowing many images to freely share their layers locally even if they didn’t come from the same build\cite{dockerlayer}. 

However, due to the active development of Docker platform and its relatively young ages of history, there is no available articles illustrating how those addressable images worked by those SHA256 hash. 
% original and cache IDs are used and maintained throughout the life cycle of one container, as well as the life cycle of the underlying docker daemon. 

By investigating into the source code of Docker as well as its storage drivers, we find out how Docker platform addresses different storage layers and propose our methods to leverage those layers in the live migration of Docker containers.
% keeps track of those mapping relationship. 

For example, we find that 
if the same image is downloaded from the same build on the cloud, Docker will mapping the layers' original layer IDs to a new secure content hash, called `cache ID'. Every image layer's original ID will be replaced with its cache ID. From then on, Docker daemon will address the image layer by this cache ID when it creates, starts, stops, or checkpoints/restores a container. 


\smallbreak
\subsubsection{ID Matching Between Docker Host} \label{intro:idMatching}
~\smallbreak

As we discussed above, 
the image layer on the cloud will have different cache IDs when downloaded to different Docker host. 
% It becomes a challenge to resolve those addressable IDs when we migrate a container between different hosts. 
Therefore, if we want to share the common image layers between different Docker hosts, we need to resolve the image layers mapping on different Docker host. 

In order to do this, on each Docker host we map back its cache IDs to their original layer IDs by querying the Docker image database. 
% that have the same origin on two different Docker host.
Then we could match the image layers by its original IDs instead of the cache IDs exposed to the local Docker daemons. More detail could be found in section \ref{idremapping}.

% In order to do this, we then need to generate a new image layer stack with different cache IDs but actually have the same content inside since they have the same original layer IDs. 


% In the Docker's root system, the /usr/lib/docker/ by default (or /usr/lib/docker/0.0/ for docker 1.10 version), 

\smallbreak
\subsubsection{Docker's Storage Driver Backend}
~\smallbreak

Docker delegate the task of managing the container's layered file system to its pluggable storage driver. 
The Docker storage driver is designed to provide a single unified view from a stack of image layers.
Users can choose different kinds of storage driver that are best for their environment and particular use-cases. 

In order to understand some inner details of how those addressable images work, we investigate the source code of Docker system along with one of its most popular storage driver, AUFS. We will show our findings inside the structure of the Docker's storage systems in this paper. For other storage drivers like Btrfs, Device Mapper, overlay, and ZFS, they implement the management of image layers and the \textit{container layer} in their own unique ways, but our framework could also be extended to those drivers. Due to limited time and space, we only conduct experiments on AUFS. The following section will discuss the inner details about our findings inside the Docker's AUFS storage driver.

% TODO: more details in images management, including the mapping details. Draw a docker root directory structure Figure

% \paragraph{AUFS storage}\mbox{}\\

\subsection{AUFS Storage: A Case Study}
The default storage driver on Docker is AUFS. Therefore, we take this as an example to introduce the layered images management.

AUFS storage driver implements Docker image layers by a union mount system. Union Mount is a way of combining numerous directories into one directory that looks like it contains the content from all the them \cite{aufs}. Using Union filesystems merge all the files for each image layer together and presents them as one single read-only directory at the union mount point. If there are duplicate files in different layers, only the file on the higher level layer is available.

AUFS driver has three directories: layers, diff, and mnt. `layers' directory stores the metadata of how image layers are stacked together. `diff' directory stores the exact data inside each layers. `mnt' directory stores the mount points for the root file system of all the containers. 

%  https://github.com/docker/docker/blob/b248de7e332b6e67b08a8981f68060e6ae629ccf/daemon/graphdriver/aufs/aufs.go
% /*
% aufs driver directory structure
%   .
%   ├── layers // Metadata of layers
%   │   ├── 1
%   │   ├── 2
%   │   └── 3
%   ├── diff  // Content of the layer
%   │   ├── 1  // Contains layers that need to be mounted for the id
%   │   ├── 2
%   │   └── 3
%   └── mnt    // Mount points for the rw layers to be mounted
%       ├── 1
%       ├── 2
%       └── 3
% */

 \tikzset{/forest,
    % symlink/.append style={
    %   opacity=.25,
    %   text opacity=.5,
    %   before drawing tree={
    %     {tikz+={\draw [thick, -{>[]}] (!#1.west) ++(4pt,-1.5pt) arc (315:120:5pt);}}
    %   },
    % },
    % file/.append style={
    file/.style={
        % minimum width=0.5cm,
        % minimum height=0.5cm,
        % parent anchor=south,
        % child anchor=north,
        % sibling distance = 4cm,
        % dogeared,
        % fill=white!50,
        % draw,
        trapezium,
        trapezium angle=0,
        rounded corners=2pt,
        draw,
        fill=white!50,
    },
   }
  
  
\begin{figure*}
\centering
% http://tex.stackexchange.com/questions/5073/making-a-simple-directory-tree
\begin{forest}
for tree={
    minimum width=0.5cm,
    minimum height=0.5cm,
    parent anchor=south,
    child anchor=north,
    trapezium,
    trapezium angle=0,
    rounded corners=2pt,
    draw,
    fill=blue!50,
    % % http://tex.stackexchange.com/questions/278708/center-root-of-forest-tree
    % if level=0{ 
    % parent anchor=south,
    % child anchor=north,
    % align=center,
    % l=1cm,
    % fill=white,
    % minimum width=\linewidth,
    % inner xsep=0pt,
    % outer xsep=0pt,
    % }{},
    % myfolder,
}
% [, phantom, s sep = 1cm
[/var/lib/docker/0.0/
[containers
    [<conID>
        [config.v2.json, file]
        % [resolve.conf, file]
        [..., file]
    ]
    % [<conID1>
    %     % [config.v2.json, file]
    %     % % [hostname, file]
    %     % [..., file]
    % ]
]   
[aufs/
    [mnt/
        [<rootfs ID>/
            % [/boot/
            %     [...]
            % ]
            [/etc/
                [hostname, file]
                [hosts, file]
            ]
            [/home/
                % [username/
                %     [.bashrc, file]
                %  [...]
                % ]
            ]
            % [/.../]
        ]
    ]
    [layers/
        [<rootfs ID>, file]
    ]
    [diff/
        [<layer ID1>/
            [/etc/
                [hostname, file]
            ]
            % [/.../]
        ]
        [<layerID2>/
            [/etc/
                [hostname, file]
                [hosts, file]
            ]
        ]
    ]
]
[image/aufs/layerdb/sha256
    % [aufs
        % [layerdb
            % [mounts
            %     % [<conID>]
            % ]
            % [sha256
                [<O-layerID>
                    [cache-id, file]
                    % [diff,file]
                    [parent,file]
                    [..., file]
                ]
            % ]
        % ]
        % [imagedb]
    % ]
]
]
\end{forest}

\caption{Docker Layered File System Structure Based on AUFS Storage Driver}
\label{fig:aufs}
\end{figure*}


%  https://github.com/docker/docker/blob/b248de7e332b6e67b08a8981f68060e6ae629ccf/daemon/graphdriver/aufs/aufs.go
% /*
% aufs driver directory structure
%   .
%   ├── layers // Metadata of layers
%   │   ├── 1
%   │   ├── 2
%   │   └── 3
%   ├── diff  // Content of the layer
%   │   ├── 1  // Contains layers that need to be mounted for the id
%   │   ├── 2
%   │   └── 3
%   └── mnt    // Mount points for the rw layers to be mounted
%       ├── 1
%       ├── 2
%       └── 3
% */

\begin{figure}
\centering
% http://tex.stackexchange.com/questions/5073/making-a-simple-directory-tree
\begin{forest}
for tree={
    minimum width=0.5cm,
    minimum height=0.5cm,
    parent anchor=south,
    child anchor=north,
    trapezium,
    trapezium angle=0,
    rounded corners=2pt,
    draw,
    fill=blue!50,
    % % http://tex.stackexchange.com/questions/278708/center-root-of-forest-tree
    % if level=0{ 
    % parent anchor=south,
    % child anchor=north,
    % align=center,
    % l=1cm,
    % fill=white,
    % minimum width=\linewidth,
    % inner xsep=0pt,
    % outer xsep=0pt,
    % }{},
    % myfolder,
}
% [, phantom, s sep = 1cm
[/var/run/docker/execdriver/native/
    [<conID1>
        [state.json, file]
    ]
    [<conID2>
        [state.json, file]
    ]
]
% ]
\end{forest}

\caption[Caption for LOF]{Runtime Data for Containers\protect\footnotemark 
% -dev, the latest Docker (version 17.04.0-ce, build 4845c56) has this directory changed to \textit{/var/run/docker/libcontainerd/containerd}}
% \footnote{This is for Docker 1.10\-dev, the latest Docker (version 17.04.0\-ce, build 4845c56) has this directory changed to \textit{/var/run/docker/libcontainerd/containerd}}
}
\label{fig:aufs-runtime}
\end{figure}

% \footnotetext{This is for Docker 1.10-dev, the latest Docker (version 17.04.0-ce, build 4845c56) has this directory changed to \textit{/var/run/docker/libcontainerd/containerd}}

Figure~\ref{fig:aufs} shows the Docker storage structure based on AUFS driver. White box stands for a file and blue box stands for a directory. Since all directories share the same parent dir \textit{/var/lib/docker/0.0}, we will use `.' to represent this common directory in the following sections. 


\smallbreak
\subsubsection{Container's Image Layer Stack List}
~\smallbreak
% \smallbreak  \subsubsection{Container's Image Layer Stack List:}

We know that each Docker image contains several image layers. Those image layers are addressed by their SHA256 content hash IDs. Each Docker image has a list that stores all these layer IDs in the order of how they stacked from top to bottom.
The file \textit{./aufs/layers/<rootfs ID>-init}  in Figure \ref{fig:aufs} shows the path of this file inside the AUFS storage structure. This file stores a list of SHA256 IDs of all image layers that will be bind mounted together as the container's root file system. For example, for one container OpenFace with rootfs ID of 
\textit{
febfb1642ebeb25857bf2a9c558bf695
\footnotetext{This is for Docker 1.10-dev, the latest Docker (version 17.04.0-ce, build 4845c56) has the runtime data directory changed to \textit{/var/run/docker/libcontainerd/containerd/}}
\footnote{SHA256 ID has 64 hexadecimal characters, here we truncate it to 32 hexadecimal characters in order to save space}
}, it's stack list file\textit{./aufs/ layers/ febfb1642ebeb25857bf2a9c558bf695}

There is another file \textit{./aufs/layers/<rootfs ID>} is the file stores this list of image layer IDs for that Docker image as well as the ID of the writable \textit{container layer}.  The \textit{container layer}'s ID is used to address the thin writable layer for that container. This \textit{container layer} ID is the same as the <rootfs ID>. 
When the Docker daemon start or restore a container, it will refer to those two files to get a list of all underlying Docker image layer IDs and the \textit{container layer} ID. Then it will resolve those addressable IDs and union mount all those layer stacks together with its thin writable \textit{container layer}. After this, the container will get the union mount of its root file system view under its root mount point. 
Now we see this file behaves like an important handler for all the union file systems for the container. If this file is missing, one container will not be able to union mount the layered file system.

% --------------

\smallbreak 
\subsubsection{Image Layer Content Directory}
~\smallbreak
Now we have the mount point, and a list of image layer IDs as the addresses for the storage layers. But before a container can start, the Docker daemon needs to know where to find the files inside each image layers. Here comes the directory of
 \textit{./aufs/diff/<layer ID>/}, which stores all the image layer contents from one specific layer identified by a local image \textit{<layer ID>}. If \textit{<layer ID>} is the same as \textit{<rootfs ID>} of one container, then this directory is where the content of \textit{container layer} stores, i.e. the container's thin writable layer. 

\smallbreak 
\subsubsection{Unified Mount Point} 
~\smallbreak
The directory \textit{./aufs/mnt/<rootfs ID>/ }  is the root file system mount point of the container. All the image layers are union mounted to this folder and provide a whole file system view for the container. For example, as shown in Figure \ref{fig:aufs}, when a container is created based on a Linux image, its mount point will contain the root directory contents like \textit{/usr/, /home/, /boot/, etc. }. All those directories are mounted from its underlying layered images. 
Since this directory is a mount point for a running container's file system, it will be only available when the container is running. If the container stops running, all the image layers will be unmounted from this mount point. So it will become an empty directory.

Here, the root file system ID, <rootfs ID>, is also an image layer cache ID of container's thin writable \textit{container layer}. 

\smallbreak  
\subsubsection{Layer ID Mapping}  \label{intro:aufs:layerIDMapping}
~\smallbreak
Until now, the layer IDs we have discussed above are just local SHA256 IDs, or the so called cache IDs, which are generated dynamically when each image layer is downloaded by `docker pull' command. As we discussed above, from then on, Docker daemon will address the image layer use the cache ID instead of its original layer ID.

We find the Docker storage system maintains a mapping relationship between the original layer IDs and its cache IDs. All the cached IDs of image layers are stored in the \textit{/image/aufs/layerdb/sha256} directory.
For example, the file \textit{./image/aufs/layerdb/sha256/<O-layerID>/cache-id} shown in Figure~\ref{fig:aufs} stores the cache ID of the image with original ID <O-layerID>. For example, if a string of \textit{
fac86d61dfe33f821e8d0e7660473381} is stored in a file path \textit{./image/ aufs/layerdb/sha256/6384c447ddd6cd859f9be3b53f8b015c/cache-id}, this means there is an image layer with an original ID of \textit{
6384c447dd-d6cd859f9be3b53f8b015c} and it's cache ID is mapped to \textit{
fac86d61df-e33f821e8d0e7660473381}.


% However, due to the active development of Docker community and its relatively young ages of history, there is no available articles illustrating how those original and cache IDs are used and maintained throughout the life cycle of one container, as well as the life cycle of the underlying docker daemon. By investigating into the source code of Docker as well as its storage system, we find out how Docker platform keeps track of those mapping relationship. 

\smallbreak  
\subsubsection{Container Configuration and Runtime State}
~\smallbreak

Finally, there are directories storing the configuration files and runtime data. Figure~\ref{fig:aufs-runtime} shows the runtime data directory stored for each containers. For one container with ID of \textit{<conID>}, there will be one JSON (JavaScript Object Notation)  file \textit{state.json} that stores the run time state of the container. For example, it stores the init pid of the containers' processes with key ``\textit{init\_process\_pid}'', the root file system mount point path with key ``\textit{rootfs}'', as well as the runtime cgroup and namespace meta data, etc.. 

Along with the runtime data directory, there is another directory inside Docker root directory (\textit{/var/lib/docker/0.0}) that stores the configuration data for each container: \textit{./containers/<conID>/} as shown in Figure~\ref{fig:aufs}. For example, \textit{ config.v2.json} file stores the container's creation time, the command that was run when creating the container, etc..


% TODO: more details in AUFS. Draw an aufs driver directory structure Figure

%   ├── layers // Metadata of layers
%   │   ├── 1
%   │   ├── 2
%   │   └── 3
%   ├── diff  // Content of the layer
%   │   ├── 1  // Contains layers that need to be mounted for the id
%   │   ├── 2
%   │   └── 3
%   └── mnt    // Mount points for the rw layers to be mounted
%       ├── 1
%       ├── 2
%       └── 3




\subsection{Docker Container Live Migration in Practice }\label{migpractice}

Although there is no official live migration tools for Docker containers yet, many enthusiastic developers have composed some customized tool for some versions of Docker platform, which shows great feasibility of Docker containers' live migration. For example, Ross Boucher's fork \cite{boucherPhaul} of P.haul
% \cite{phaul}
can be used for live migration of Docker containers for Docker 1.10-dev. However, after investigation, we found it doesn't leverage the layered images of Docker containers. It simply transfers all the file system of that container from source server to the target server, where the files are actually composed by all the image layers of that container. This made the tool not feasible and will crash the layered file system on the target Docker platform. Firstly, transferring all the image layers at once made the live migration process pretty slow for applications that has big file systems. 

To verify our claim, we conducted experiments to live migrate containers through different network connections using this tool. We experiments on one simple container, busybox, and two applications for edge server offloading: OpenFace and ?.  Busybox is a stripped-down Unix tools in a single executable file. It has very tiny file system inside the container. OpenFace\cite{openface2016} is an application that will ship images from mobile device to the edge server, then execute the face recognition algorithm on server, and finally sending back a text string of the name of the person to the mobile device. The file system is huge for this container, which is approxmitely $2$ gigabytes.

Table~\ref{table_samehost} shows the live migration could be done within 10 seconds for busybox and with 30 seconds for OpenFace. The two nodes are two virtual machines on the same physical host. The network bandwidth is high with just an RTT of 0.4 milliseconds, which allows to transfer 2.17 GB data within a short time.



\begin{table}[!t]
% increase table row spacing, adjust to taste
\renewcommand{\arraystretch}{2.1}
% if using array.sty, it might be a good idea to tweak the value of
% \extrarowheight as needed to properly center the text within the cells
\centering
\begin{tabular}{|l|l|l|l|l|l|}
%  \begin{tabular}{|M|M|M|l|l|l|}
\hline
App & \pbox{2cm}{Total \\ time}  & \pbox{2cm}{Down\\ time} & \pbox{2cm}{ FS Size } & \pbox{2cm}{ Total Size } & \pbox{2cm}{ network \\ RTT } \\ \hline 
Busybox & $7.54$ $s$ & $3.49$ $s$ & 140 KB & 290KB &  0.401 ms \\\hline
OpenFace & $26.19$ $s$ & $5.02$ $s$ & 2.0 GB & 2.17GB &  0.401 ms \\\hline
\end{tabular}

\caption{Docker Container Migration Time (between two VMs on the same host machine)}
\label{table_samehost}

\end{table}

\begin{table}[!t]
% \renewcommand{\arraystretch}{2.1}
\centering
\begin{tabular}{|M{1.2cm}|M{0.9cm}|M{0.9cm}|M{1.4cm}|M{1.1cm}|M{0.9cm}|}
\hline
% Container & Total time  & Down time &  FS transferred & \pbox{1cm}{ Total\\ tranferred} & \pbox{1cm}{ network \\ RTT } \\ \hline 
App & Total time  & Down time &  File System & Total & network bandwidth \\ \hline 
busybox & $133.11 s$ & $9 s$ & 140 KB & 290KB & 5.41 ms\\\hline
openface & $\sim3200s$ & $153.82s$ & 2.0G & 2.17G & 5.41 ms\\\hline
\end{tabular}

\caption{Docker Container Migration Time (between two different hosts through Wireless LAN)}
\label{table_wireless}

\end{table}



However, when we live migrate containers between two physical host on the same LAN network, the performance is reduced dramatically. Table~\ref{table_wireless} shows that the migration of the busybox container will take 133.11 seconds with just 152 kilobytes of transferred size. And for the heavy offloading application of OpenFace, it cost about 3200 seconds with transferred sized of more than 2 Gigabytes. After investigation into its detailed steps of live migration, we found it didn't make use of the layered infrastructure. Instead of transfering only the thin container layer modified inside the container, it transfers the whole file system including all the stacked image layers of the container, which is definitely not a proper way to migrate Docker containers.  This behavior would cause severe problems from several aspects:
% \begin{enumerate}[series = tobecont , itemjoin = \quad]
\begin{enumerate}[series = tobecont]
    \item It will corrupt the layered file system inside the container after restored  on the target server. The tools just transfer the whole file system into one directory on the destination server and mount it as root directory for the container. After restored on the target host, the container no longer maintains its layered image stacks as on the source node. 
    \item It dramatically reduces the efficiency of live migration. The transferring size of live migrating a container with its whole file system would just worse than a mature VM live migration. Because even live migration could avoid transferring the whole file system by sharing the base VM images as shown in \ref{motivation:vmhandoff}. 
\end{enumerate}
Therefore, we need a new tool for live migrating the Docker containers that could leverage the layerred file systems and avoid the transmisssion of common image layer stack by just transferring the \textit{container layer}.

% \subsection{Docker Layered Images }

% As a container engine, Docker is increasingly popular in industrial cloud platform. It serves as an composing engine for Linux containers, where applications are running in an isolated environment based on OS-level virtualization. Docker enables layered storage inside containers, which allows fast packaging and shipping of any application as a lightweight container. Each Docker image references a list of read-only layers that represent filesystem differences. Layers are stacked on top of each other to form a base for a container’s root filesystem \cite{dockerlayer}. 
% % \cite{https://docs.docker.com/engine/userguide/storagedriver/imagesandcontainers/#images-and-layers}

% When a new container is started, a new, thin, writable layer, called \textit{``container layer''} is created on top of the underlying stack. All changes made to the running container are written to this thin layer. 
% % It greatly simplifies the process of setting environment for any software platform. 

% This layered storage allows fast live migration of containers' run-time states without having to transfer the system and application base images. With the cloud storage of container images (like in DockerHub), all the container images are available anywhere across the Internet. Therefore, before any live migration starts, any edge server has the chance to download the system and application images as the base images stack for the container to run on. 
% During the migration, we only need to transfer the run-time memory states and the thin container layer on top of the Docker images stack. 


% Apparently, the live migration of the Docker containers seems to be a better choice than the virtual machine based approaches we introduced above. The layered storage in Docker infrastructure enlightens a great opportunity for the service hand-off in edge computing. If the live migration of containers can be done very fast, we can have the service hand-off built ontop of it to get high speed hand-off.
