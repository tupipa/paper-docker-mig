
In this section, we evaluate the performance of the system for different workloads under different network and computing conditions. Specifically, we evaluate the offloading service handoff system by showing in what extent the total hand-off time and total transferred size are reduced.

In our experiment, we set up three migration scenarios
% \footnote{TODO:need new applications of experiments}: 
\begin{enumerate}
    \item Emulated bandwidth:
    Docker Hosts are two virtual machines on the same Desktop server. On the source virtual machine, we use Linux Traffic Control (tc\cite{tc} )  tool to control network traffic. Limited the bandwidth to 5Mbps, 15Mbps, 20Mbps, 25Mpbs and 30Mpbs. The latency is set to a fixed 50ms \footnote{some cites here}.
    Docker containers are migrated from the Docker host on source VM to the Docker host on another VM.
    
    % \item Laptop-Wireless-WAN:
    % The migration source host is a virtual machine on the Laptop, another is the virtual machine on the server as in 1). Both virtual machines are running the same version of Docker.  They are connected through WAN network with about 2 miles geological distance. Each virtual machines is running within its own LAN network. The virtual machine on Laptop connects to the internet through wireless adapter. The virtual machine on the server is connected to the Internet via Ethernet. 
\end{enumerate}

The desktop server is equipped with Intel$^{R}$ Core$^{TM}$ i3-6100 Processor (3.70GHz, 2 cores, 4 threads) and 16GB DDR4 memory. Two virtual machines are running with 2 vcpus and 4GB memory each.
% The laptop is with Intel$^{R}$ Core$^{TM}$ Duo T6570 (2.2GHz, 2 cores) and 2GB DDR2 memory. Both the machines and virtual machines are running Ubuntu 16.04 LTS as the operating system.
The laptop is with Intel$^{R}$ Core$^{TM}$ i5-2450M CPU (2.5GHz, 2 cores, 4 threads) and 4GB DDR3 memory. Both the machines and virtual machines are running Ubuntu 16.04 LTS as the operating system. Docker version is 1.10-dev which is built with experimental feature enabled.




TODO one large table here:

\begin{table}[h]
\centering
\begin{tabular}{|P{3.5em}|P{2em}|P{3em}|P{2.8em}|P{3.5em}|P{3.5em}|}
\hline
\multicolumn{2}{|P{5em}|}{Bandwidth (Mbps)} &  Handoff Time(s) & Down Time(s) & Pre-Transfer Size & Final-Transfer Size \\                \hline
\multirow{13}{*}{Busybox} 
 &5     &3.90  & 3.23  & & c\\
 &15    &X  &   & c &c \\
 &20    &x  &x  & c &c \\
 &25    &x  &x  & c &c \\
 &30    &x  &x  & c &c \\
 &35    &x  &x  & c &c \\
 &40    &x  &x  & c &c \\
 &45    &x  &x  & c &c \\
 &50    &x  &x  & c &c \\
 &100    &x  &x  & c &c \\
 &200    &x  &x  & c &c \\
 &500    &x  &x  & c &c \\
 &1000    &x  &x  & c &c \\\cline{1-6}
\multirow{13}{*}{Busybox}
 &5     &n  &   & c &c \\
 &15    &X  &   & c &c \\
 &20    &x  &x  & c &c \\
 &25    &x  &x  & c &c \\
 &30    &x  &x  & c &c \\
 &35    &x  &x  & c &c \\
 &40    &x  &x  & c &c \\
 &45    &x  &x  & c &c \\
 &50    &x  &x  & c &c \\
 &100    &x  &x  & c &c \\
 &200    &x  &x  & c &c \\
 &500    &x  &x  & c &c \\
 &1000    &x  &x  & c &c \\\cline{1-6}
\end{tabular}
    \caption{
        Average of 10 runs and relative standard deviations (RSDs, in parentheses) are reported. 
        Handoff time is from the time the server receives migration request until migration done; 
        The down time is the time from the container is checkpointed to the container is restored on the target.
    }
    \label{tab:perform}
\end{table}

--------------------------------------

bandwidth/app   migration time / down time  /  FS size / Image size

5Mbps, 50ms

 \quad busybox
    
 \quad   face
    
  \quad  ...
    
10Mbps, 50ms

 \quad busybox
    
 \quad   face

  \quad  ...

15Mbps, 50ms

 \quad busybox
    
 \quad   face

  \quad  ...

25Mbps, 50ms

 \quad busybox
    
 \quad   face

  \quad  ...


30Mbps, 50ms

==========================================

% \begin{table}[!t]
% \centering

% \caption{Optimized using both layered images and compression with 5 Mbps bandwidth and 50 ms latency}
% \label{table_samehost_opt}


% % Some packages, such as MDW tools, offer better commands for making tables
% % than the plain LaTeX2e tabular which is used here.
% % \begin{tabular}{|l|l|l|l|l|l|}
% \begin{tabular}{|M{1.2cm}|M{0.9cm}|M{0.9cm}|M{1.3cm}|M{1.1cm}|
% % M{1.0cm}|
% }
% \hline
% Container & {Total time}  & {Down time} & { FS size} & { Mem size } 
% % & { Total Size }
% \\ \hline
% Busybox & $4.13s$ & $4.10s$ & $14KB$ & $29KB$ \\\hline
% OpenFace & $22.14 s$ & $22.00s$ & $100KB$ & $102.1MB$ \\\hline
% \end{tabular}
% \end{table}

% \begin{table}[!t]
% % increase table row spacing, adjust to taste
% % \renewcommand{\arraystretch}{2.5}
% % if using array.sty, it might be a good idea to tweak the value of
% % \extrarowheight as needed to properly center the text within the cells
% \centering

% \caption{Optimized using both layered images and compression with 5 Mbps bandwidth and 50 ms latency}
% \label{table_ethernet_opt}

% % Some packages, such as MDW tools, offer better commands for making tables
% % than the plain LaTeX2e tabular which is used here.
% % \begin{tabular}{|l|l|l|l|l|l|}
% \begin{tabular}{|M{1.2cm}|M{0.9cm}|M{0.9cm}|M{1.3cm}|M{1.1cm}|
% % M{1.0cm}|
% }
% \hline
% Container & {Total time}  & {Down time} & { FS size} & { Mem size } 
% % & { Total Size }
% \\ \hline 
% Busybox & $5.24s$ & $4.4s$ & $14KB$ & $29KB$ \\\hline
% OpenFace & $29.93s$ & $?s$ & $101KB$ & $102.1MB$ \\\hline
% \end{tabular}

% \end{table}


% \begin{table}[!t]
% % increase table row spacing, adjust to taste
% % \renewcommand{\arraystretch}{2.5}
% % if using array.sty, it might be a good idea to tweak the value of
% % \extrarowheight as needed to properly center the text within the cells
% \centering

% \caption{Optimized using both layered images and compression with 5 Mbps bandwidth and 50 ms latency}
% \label{table_wireless_opt}

% % Some packages, such as MDW tools, offer better commands for making tables
% % than the plain LaTeX2e tabular which is used here.
% % \begin{tabular}{|l|l|l|l|l|l|}
% \begin{tabular}{|M{1.2cm}|M{0.9cm}|M{0.9cm}|M{1.3cm}|M{1.1cm}|
% % M{1.0cm}|
% }
% \hline
% Container & {Total time}  & {Down time} & { FS size} & { Mem size } 
% % & { Total Size }
% \\ \hline
% Busybox & $7.12s$ & $6.80s$ & $14KB$ & $29KB$ \\\hline
% OpenFace & $60.23s$ & $60.20s$ & $101KB$ & $102.1MB$\\\hline
% \end{tabular}

% \end{table}




\begin{figure}[tb!]
    \centering
    \includegraphics[width=0.48\textwidth]
    % {figure/test-tarssh.eps}
    {figure/evaluations/compare-50ms-6ms-busybox.eps}
    \caption{Busybox: Comparison of Migration Time Under Latency of 50ms and 6ms. }
    \label{fig:busylatency}
\end{figure}

\begin{figure}[tb!]
    \centering
    \includegraphics[width=0.48\textwidth]
    % {figure/test-tarssh.eps}
    {figure/evaluations/compare-50ms-6ms-face.eps}
    
    \caption{OpenFace: Comparison of Migration Time Under Latency of 50ms and 6ms. 
    }
    \label{fig:facelatency}
\end{figure}



\begin{figure}[tb!]
    \centering
    \includegraphics[width=0.48\textwidth]
    % {figure/test-tarssh.eps}
    {figure/evaluations/compare-pipe-nopipe-busybox.eps}
    \caption{Busybox: Time of Container Migration Stages with and without Pipelined Processing. }
    \label{fig:busypipe}
\end{figure}

\begin{figure}[tb!]
    \centering
    \includegraphics[width=0.48\textwidth]
    % {figure/test-tarssh.eps}
    {figure/evaluations/compare-pipe-nopipe-face.eps}
    
    \caption{OpenFace: Time of Container Migration Stages with and without Pipelined Processing
    % Pipelined Processing of ImgDiff, ImgSend, and MemApply 
    }
    \label{fig:facepipe}
\end{figure}


% \begin{figure}[tb!]
%     \centering
%     \includegraphics[width=0.48\textwidth]
%     % {figure/test-tarssh.eps}
%     {figure/evaluations/compare-pipe-nopipe-busybox.eps}
%     \caption{Busybox: Time of Container Migration Stages with and without pipeline. }
%     \label{fig:nopipe}
% \end{figure}

% \begin{figure}[tb!]
%     \centering
%     \includegraphics[width=0.48\textwidth]
%     % {figure/test-tarssh.eps}
%     {figure/evaluations/compare-pipe-nopipe-face.eps}
    
%     \caption{Time of Container Migration Stages with and without Pipelined Processing
%     % Pipelined Processing of ImgDiff, ImgSend, and MemApply 
%     }
%     \label{fig:pipe}
% \end{figure}


% TODO :
% In order to make the result more comparable to others' work and avoid the effect of certain hardwares, we need to use Linux Traffic Control (tc\cite{tc} )  tool to control network traffic.
