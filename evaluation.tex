
In this section, we evaluate the performance of our system under different workloads and different network conditions. 
Specifically, we evaluate offloading with our service handoff system by measuring the degree to which total hand-off time and total transferred size are reduced.

\subsection{Set-Up and Benchmark Workloads}

Our test bed is built on a desktop server, which is equipped with Intel Core i3-6100 Processor (3.70GHz, 2 cores, 4 threads), and 16GB DDR4 memory. Two virtual machines are running with 2 vcpus and 4GB memory each.
% The laptop is with Intel$^{R}$ Core$^{TM}$ Duo T6570 (2.2GHz, 2 cores) and 2GB DDR2 memory. Both the machines and virtual machines are running Ubuntu 16.04 LTS as the operating system.
A laptop acts as a client, this has Intel Core i5-2450M CPU (2.5GHz, 2 cores, 4 threads) and 4GB DDR3 memory. 
The hosts and virtual machines are running Ubuntu 16.04 LTS as the operating system. Docker version is 1.10-dev which is built with experimental feature enabled.

In our experiment, we set up migration scenarios using 
emulated bandwidth on two virtual machines each running a Docker instance.
Docker containers are migrated from the Docker host on source VM to the Docker host on another VM. 
Two virtual machines run on the same physical server. 

We use Linux Traffic Control (tc\cite{tc} )  tool to control network traffic. In order to test our system under WANs , we emulated low bandwidth ranging from 5Mbps to 45Mbps. According to the average bandwidth observed on the Internet\cite{internet2014state}, we set  latency at a fixed 50ms to emulate the WAN environment for edge computing.

Since edge computing environments can also be adapted to the LAN network such as a university, smart homes, or other community, it usually can reach higher bandwidth as well as lower latency. Therefore, we also tested several higher bandwidths, ranging from 50Mpbs to 500Mpbs. The latency is set to 6ms which is an average latency from the author's university LAN.

For the offloading workloads, we use Busybox as a simple workload to show the functionality of the system as well as the non-avoidable system overhead when processing container migration. In order to show the performance of offloading service handoff regarding real world applications, we chose OpenFace to provides a sample workload.

% \subsubsection{Busybox}



% \subsubsection{OpenFace
% % : A Realtime Wearable Cognitive Application
% }

\subsection{Dirty Memory \& Experimental Findings}

\begin{figure}[b]
\begin{subfigure}[b]{0.3\textwidth}
    % \centering
    \includegraphics[width=\linewidth]
    % {figure/test-tarssh.eps}
    {figure/dumps-busybox/busybox-d34455dump_iter_11.eps}
    \caption{Memory Snapshot Size}
    \label{fig:busydump}
\end{subfigure}

\begin{subfigure}[b]{0.23\textwidth}
    % \centering
    \includegraphics[width=\linewidth]
    % {figure/test-tarssh.eps}
    {figure/dumps-busybox/busybox-d34455-iter-11.eps}
    \caption{Dirty Memory, original}
    \label{fig:busydiff-ori}
\end{subfigure}
\begin{subfigure}[b]{0.23\textwidth}
    % \centering
    \includegraphics[width=\linewidth]
    % {figure/test-tarssh.eps}
    {figure/dumps-busybox/busybox-d34455-iter-11-2.eps}
    \caption{Dirty Memory, adjacent}
    \label{fig:busydiff-adj}
\end{subfigure}
% \begin{subfigure}[b]{0.23\textwidth}
%     % \centering
%     \includegraphics[width=\linewidth]
%     % {figure/test-tarssh.eps}
%     {figure/dumps-busybox/busybox-d34455diffdiff_ori11.eps}
%     \caption{Dirty Difference, original}
%     \label{fig:busydiffdiff-adj}
% \end{subfigure}
% \begin{subfigure}[b]{0.23\textwidth}
%     % \centering
%     \includegraphics[width=\linewidth]
%     % {figure/test-tarssh.eps}
%     {figure/dumps-busybox/busybox-d34455diffdiff_adj11.eps}
%     \caption{Dirty Difference, adjacent}
%     \label{fig:busydiffdiff-ori}
% \end{subfigure}

\caption{ Dirty Memory Size Analysis for Busybox. 
Figure \ref{fig:busydump} shows checkpointed memory size for total 11 dumps.
% Container is checkpointed and restored immediately in order to simulate the --leave-running options of checkpoint command.
Figure \ref{fig:busydiff-ori} shows dirty memory size beween each of dump 1~10 and the original dump 0.
Figure \ref{fig:busydiff-adj} shows dirty memory size between two adjacent dumps.
% Figure \label{fig:busydiffdiff-adj} shows the difference of dirty memories between three adjacent dumps.
% Figure \ref{fig:busydiffdiff-ori} shows the difference of dirty memories compared to the original dirty memory from dump 0 to dump 1. 
} 
\label{fig:busydiff}
\end{figure}



\begin{figure}[b]
\begin{subfigure}[b]{0.23\textwidth}
    % \centering
    \includegraphics[width=\linewidth]
    % {figure/test-tarssh.eps}
    {figure/dumps/face-518744dump_iter_11.eps}
    \caption{Memory Snapshot Size}
    \label{fig:dump}
\end{subfigure}
\begin{subfigure}[b]{0.23\textwidth}
    % \centering
    \includegraphics[width=\linewidth]
    % {figure/test-tarssh.eps}
    {figure/dumps/face-518744-iter-11.eps}
    \caption{Dirty Memory, original}
    \label{fig:diff-ori}
\end{subfigure}
\begin{subfigure}[b]{0.23\textwidth}
    % \centering
    \includegraphics[width=\linewidth]
    % {figure/test-tarssh.eps}
    {figure/dumps/face-518744-iter-11-2.eps}
    \caption{Dirty Memory, adjacent}
    \label{fig:diff-adj}
\end{subfigure}

\begin{subfigure}[b]{0.23\textwidth}
    % \centering
    \includegraphics[width=\linewidth]
    % {figure/test-tarssh.eps}
    {figure/dumps/face-518744diffdiff_ori11.eps}
    \caption{Dirty Difference, original}
    \label{fig:diffdiff-adj}
\end{subfigure}
\begin{subfigure}[b]{0.23\textwidth}
    % \centering
    \includegraphics[width=\linewidth]
    % {figure/test-tarssh.eps}
    {figure/dumps/face-518744diffdiff_adj11.eps}
    \caption{Dirty Difference, adjacent}
    \label{fig:diffdiff-ori}
\end{subfigure}

\caption{ Dirty Memory Size Analysis for OpenFace. 
The data is collected from 11 dumps of the running container of OpenFace, labeled from dump 0 to 10. Memory is dumped every 10 seconds. Container continues to run during the first 10 dumps, and stops after 11th dump.
Figure \ref{fig:dump} shows checkpointed memory size for total 11 dumps.
% Container is checkpointed and restored immediately in order to simulate the --leave-running options of checkpoint command.
Figure \ref{fig:diff-ori} shows dirty memory size beween each of dump 1~10 and the original dump 0.
Figure \ref{fig:diff-adj} shows dirty memory size between two adjacent dumps.
Figure \label{fig:diffdiff-adj} shows the difference of dirty memories between three adjacent dumps.
Figure \ref{fig:diffdiff-ori} shows the difference of dirty memories compared to the original dirty memory from dump 0 to dump 1. } 
\end{figure}



\ref{fig:busydiff}  and 
Figure \ref{fig:facediff}
give an overview of the dump memory image sizes while running the container of Busybox and  OpenFace. 
The data is collected from 11 dumps of the running container of Busybox and OpenFace, labeled from dump 0 to 10. Memory is dumped every 10 seconds. Container continues to run during the first 10 dumps, and stops after 11th dump. From the figure we can see the memory difference (or dirty memory) is much smaller than the original memory dump. The migration could be sped up by only  transferring the memory difference. 

Furthermore, 
Figure  
\ref{fig:busydiff-ori} $\sim$
\ref{fig:busydiff-adj}  
% \ref{fig:busydiffdiff-ori}  
and 
\ref{fig:facediff-ori} $\sim$
% \ref{fig:facediffdiff-ori} 
\ref{fig:facediff-adj} 
show us the sizes of memory differences between adjacent dumps as well as the original dump. We find a feature of memory access patterns for Busybox and OpenFace application. Although their memory is continuously changing, the changes reside in a specified area: a 4KB area for Busybox and 25MB area for OpenFace. 
In this case, iterative transferring of memory difference to the target server will be waste of effort. Although this will not happen on every application, it is valuable to be considered as a special case for further research regarding the migration of containers by iterative transfer. One way to utilize those kinds of memory access patterns is by not iteratively transferring every memory difference. Instead, transfer only the last iterative difference to achieve the same end result.

\subsection{Evaluation of Pipeline Performance}


\begin{figure}[tb!]
    \centering
    \includegraphics[width=0.48\textwidth]
    % {figure/test-tarssh.eps}
    {figure/evaluations/compare-pipe-nopipe-busybox.eps}
    \caption{Busybox: Time of Container Migration Stages with and without Pipelined Processing. }
    \label{fig:busypipe}
\end{figure}

\begin{figure}[tb!]
    \centering
    \includegraphics[width=0.48\textwidth]
    % {figure/test-tarssh.eps}
    {figure/evaluations/compare-pipe-nopipe-face.eps}
    
    \caption{OpenFace: Time of Container Migration Stages with and without Pipelined Processing
    % Pipelined Processing of ImgDiff, ImgSend, and MemApply 
    }
    \label{fig:facepipe}
\end{figure}

In order to verify the effectiveness of pipelined processing, we implemented the pipeline processing of two time consuming steps: \textit{imgDiff} and \textit{imgSend}, where \textit{imgDiff} is to get memory difference and \textit{imgSend} is to send memory difference to the target during migration. 
Figure \ref{fig:busypipe} and Figure \ref{fig:facepipe} reports the timing benefits we could achieve by using pipelined processing. From the figure, we can see that, without pipelined processing, most time costs are by getting and sending the memory difference. After applying pipelined processing, we could save $5\sim 8$ seconds for OpenFace migration. Busybox also saves a certain amount of time with pipelined processing.


\subsection{Overall Performance and Comparison with State-of-the-Art}
We evaluate the total \textit{handoff time} and \textit{transferred size } during offloading service handoff under different bandwidths and network latencies. 

Table \ref{tab:all} shows an overview of the performance of our system under different network bandwidth conditions.
% 
\begin{figure}[tb!]
    \centering
\begin{subfigure}[b]{0.23\textwidth}
    % \centering
    \includegraphics[width=\linewidth]
    % {figure/test-tarssh.eps}
    {figure/eva-individual/predump-docker-pipe-busybox-stack-bar.eps}
    \caption{Busybox}
    \label{fig:allband-busy}
\end{subfigure}
\begin{subfigure}[b]{0.23\textwidth}
    % \centering
    \includegraphics[width=\linewidth]
    % {figure/test-tarssh.eps}
    {figure/eva-individual/predump-docker-pipe-face-stack-bar.eps}
    \caption{OpenFace }
    \label{fig:allband-face}
\end{subfigure}
    \caption{Time of Container Migration Under Different Bandwidths (WANs \& LANs).
    }
    \label{fig:allband}
\end{figure}

% Figure \ref{fig:allband}.
% Figure \ref{fig:allband-busy} 
% Figure \ref{fig:allband-face}



TODO one large table here:

\begin{table}[h]
\centering
\begin{tabular}{|P{3.5em}|P{2em}|P{3em}|P{2.8em}|P{3.5em}|P{3.5em}|}
\hline
\multicolumn{2}{|P{5em}|}{Bandwidth (Mbps)} &  Handoff Time(s) & Down Time(s) & Pre-Transfer Size & Final-Transfer Size \\                \hline
\multirow{13}{*}{Busybox} 
 &5     &3.90  & 3.23  & & c\\
 &15    &X  &   & c &c \\
 &20    &x  &x  & c &c \\
 &25    &x  &x  & c &c \\
 &30    &x  &x  & c &c \\
 &35    &x  &x  & c &c \\
 &40    &x  &x  & c &c \\
 &45    &x  &x  & c &c \\
 &50    &x  &x  & c &c \\
 &100    &x  &x  & c &c \\
 &200    &x  &x  & c &c \\
 &500    &x  &x  & c &c \\
 &1000    &x  &x  & c &c \\\cline{1-6}
\multirow{13}{*}{Busybox}
 &5     &n  &   & c &c \\
 &15    &X  &   & c &c \\
 &20    &x  &x  & c &c \\
 &25    &x  &x  & c &c \\
 &30    &x  &x  & c &c \\
 &35    &x  &x  & c &c \\
 &40    &x  &x  & c &c \\
 &45    &x  &x  & c &c \\
 &50    &x  &x  & c &c \\
 &100    &x  &x  & c &c \\
 &200    &x  &x  & c &c \\
 &500    &x  &x  & c &c \\
 &1000    &x  &x  & c &c \\\cline{1-6}
\end{tabular}
    \caption{
        Average of 10 runs and relative standard deviations (RSDs, in parentheses) are reported. 
        Handoff time is from the time the server receives migration request until migration done; 
        The down time is the time from the container is checkpointed to the container is restored on the target.
    }
    \label{tab:perform}
\end{table}

--------------------------------------

bandwidth/app   migration time / down time  /  FS size / Image size

5Mbps, 50ms

 \quad busybox
    
 \quad   face
    
  \quad  ...
    
10Mbps, 50ms

 \quad busybox
    
 \quad   face

  \quad  ...

15Mbps, 50ms

 \quad busybox
    
 \quad   face

  \quad  ...

25Mbps, 50ms

 \quad busybox
    
 \quad   face

  \quad  ...


30Mbps, 50ms

==========================================

% \begin{table}[!t]
% \centering

% \caption{Optimized using both layered images and compression with 5 Mbps bandwidth and 50 ms latency}
% \label{table_samehost_opt}


% % Some packages, such as MDW tools, offer better commands for making tables
% % than the plain LaTeX2e tabular which is used here.
% % \begin{tabular}{|l|l|l|l|l|l|}
% \begin{tabular}{|M{1.2cm}|M{0.9cm}|M{0.9cm}|M{1.3cm}|M{1.1cm}|
% % M{1.0cm}|
% }
% \hline
% Container & {Total time}  & {Down time} & { FS size} & { Mem size } 
% % & { Total Size }
% \\ \hline
% Busybox & $4.13s$ & $4.10s$ & $14KB$ & $29KB$ \\\hline
% OpenFace & $22.14 s$ & $22.00s$ & $100KB$ & $102.1MB$ \\\hline
% \end{tabular}
% \end{table}

% \begin{table}[!t]
% % increase table row spacing, adjust to taste
% % \renewcommand{\arraystretch}{2.5}
% % if using array.sty, it might be a good idea to tweak the value of
% % \extrarowheight as needed to properly center the text within the cells
% \centering

% \caption{Optimized using both layered images and compression with 5 Mbps bandwidth and 50 ms latency}
% \label{table_ethernet_opt}

% % Some packages, such as MDW tools, offer better commands for making tables
% % than the plain LaTeX2e tabular which is used here.
% % \begin{tabular}{|l|l|l|l|l|l|}
% \begin{tabular}{|M{1.2cm}|M{0.9cm}|M{0.9cm}|M{1.3cm}|M{1.1cm}|
% % M{1.0cm}|
% }
% \hline
% Container & {Total time}  & {Down time} & { FS size} & { Mem size } 
% % & { Total Size }
% \\ \hline 
% Busybox & $5.24s$ & $4.4s$ & $14KB$ & $29KB$ \\\hline
% OpenFace & $29.93s$ & $?s$ & $101KB$ & $102.1MB$ \\\hline
% \end{tabular}

% \end{table}


% \begin{table}[!t]
% % increase table row spacing, adjust to taste
% % \renewcommand{\arraystretch}{2.5}
% % if using array.sty, it might be a good idea to tweak the value of
% % \extrarowheight as needed to properly center the text within the cells
% \centering

% \caption{Optimized using both layered images and compression with 5 Mbps bandwidth and 50 ms latency}
% \label{table_wireless_opt}

% % Some packages, such as MDW tools, offer better commands for making tables
% % than the plain LaTeX2e tabular which is used here.
% % \begin{tabular}{|l|l|l|l|l|l|}
% \begin{tabular}{|M{1.2cm}|M{0.9cm}|M{0.9cm}|M{1.3cm}|M{1.1cm}|
% % M{1.0cm}|
% }
% \hline
% Container & {Total time}  & {Down time} & { FS size} & { Mem size } 
% % & { Total Size }
% \\ \hline
% Busybox & $7.12s$ & $6.80s$ & $14KB$ & $29KB$ \\\hline
% OpenFace & $60.23s$ & $60.20s$ & $101KB$ & $102.1MB$\\\hline
% \end{tabular}

% \end{table}



\textit{Handoff time}
is from the time the source host receives a migration request until the offloading container is successfully restored on the target host. The \textit{Down time} is from the time when the container is stopped on the source server to the time when the container is restored on the target server. \textit{Handoff time} and \textit{Down time} are very similar since we immediately checkpoint and stop the container once we get the migration request. 

\textit{Pre-Transfer Size} is the transferred size before \textit{handoff} starts, i.e. from stage \ref{prepare} until stage \ref{request} . \textit{Final-Transfer Size} is the transferred size during the handoff, i.e. from stage \ref{request} until the end of final stage \ref{restore}.


\begin{figure}[tb!]
    \centering
    \includegraphics[width=0.48\textwidth]
    % {figure/test-tarssh.eps}
    {figure/evaluations/compare-50ms-6ms-busybox.eps}
    \caption{Busybox: Comparison of Migration Time Under Latency of 50ms and 6ms. }
    \label{fig:busylatency}
\end{figure}

\begin{figure}[tb!]
    \centering
    \includegraphics[width=0.48\textwidth]
    % {figure/test-tarssh.eps}
    {figure/evaluations/compare-50ms-6ms-face.eps}
    
    \caption{OpenFace: Comparison of Migration Time Under Latency of 50ms and 6ms. 
    }
    \label{fig:facelatency}
\end{figure}

Figure \ref{fig:busylatency}
and
Figure \ref{fig:facelatency} shows the performance under difference network latencies of $50ms$ and $6ms$ for Busybox and OpenFace.  It shows a tiny difference when facing different latencies. This means our system is suitable for a wide range of network latencies.


From Table \ref{tab:all} and Figure \ref{fig:busylatency} , we can see the simple Busybox container can be migrated very quickly regardless the network bandwidth and latency. 

From Table \ref{tab:all} and Figure \ref{fig:facelatency} , we can see the OpenFace offloading container can be migrated within $49$ seconds under the lowest bandwidth $5$Mbps with $50$ ms latency. Compared to the work in \cite{ha2015vmhandoff}, which has a handoff time of $247$ seconds under the same network conditions. For $25$Mbps and $50$ ms latency, our system achieves $17.4$ seconds while  \cite{ha2015vmhandoff}  needs $39$ seconds. The relative standard deviations in Table \ref{tab:all} shows the robustness of our experimental result.


Therefore, our system could reduce the total handoff time by $56\%\sim 80\%$ compared to the state-of-the-art work of VM handoff \cite{ha2015vmhandoff} on edge computing platforms. 






% \subsubsection{Performance Under Different Bandwidths}

% \subsubsection{Performance Under Different Latencies}







    % \item Laptop-Wireless-WAN:
    % The migration source host is a virtual machine on the Laptop, another is the virtual machine on the server as in 1). Both virtual machines are running the same version of Docker.  They are connected through WAN network with about 2 miles geological distance. Each virtual machines is running within its own LAN network. The virtual machine on Laptop connects to the internet through wireless adapter. The virtual machine on the server is connected to the Internet via Ethernet. 
% \end{enumerate}
