%  https://github.com/docker/docker/blob/b248de7e332b6e67b08a8981f68060e6ae629ccf/daemon/graphdriver/aufs/aufs.go
% /*
% aufs driver directory structure
%   .
%   ├── layers // Metadata of layers
%   │   ├── 1
%   │   ├── 2
%   │   └── 3
%   ├── diff  // Content of the layer
%   │   ├── 1  // Contains layers that need to be mounted for the id
%   │   ├── 2
%   │   └── 3
%   └── mnt    // Mount points for the rw layers to be mounted
%       ├── 1
%       ├── 2
%       └── 3
% */

 \tikzset{/forest,
    % symlink/.append style={
    %   opacity=.25,
    %   text opacity=.5,
    %   before drawing tree={
    %     {tikz+={\draw [thick, -{>[]}] (!#1.west) ++(4pt,-1.5pt) arc (315:120:5pt);}}
    %   },
    % },
    % file/.append style={
    file/.style={
        % minimum width=0.5cm,
        % minimum height=0.5cm,
        % parent anchor=south,
        % child anchor=north,
        % sibling distance = 4cm,
        % dogeared,
        % fill=white!50,
        % draw,
        trapezium,
        trapezium angle=0,
        rounded corners=0pt,
        draw,
        fill=white!50,
    },
   }
  
  
\begin{figure*}
\centering
% http://tex.stackexchange.com/questions/5073/making-a-simple-directory-tree
\begin{forest}
for tree={
    minimum width=0.5cm,
    minimum height=0.5cm,
    parent anchor=south,
    child anchor=north,
    trapezium,
    trapezium angle=95,
    rounded corners=2pt,
    draw,
    fill=blue!50,
    % myfolder,
}
[/var/lib/docker/0.0/
[aufs/
    [mnt/
        [<rootfs ID>/
            [/boot/
                [...]
            ]
            [/usr/
                [...]
            ]
            [/home/
                [username/
                    [.bashrc, file]
                    [... , file]
                ]
            ]
            [/.../]
        ]
    ]
    [layers/
        [<rootfs ID>, file]
    ]
    [diff/
        [<layerID>/
            [/lib/
                [...]
            ]
            [/.../]
        ]
        [<layerID>/
            [/home/
                [usr/]
            ]
        ]
    ]
]
[image
    [aufs
        [layerdb
            % [mounts
            %     % [<conID>]
            % ]
            [sha256
                [<O-layerID>
                    [cache-id, file]
                    [diff,file]
                    [parent,file]
                    [..., file]
                ]
            ]
        ]
        % [imagedb]
    ]
]
[containers
    [<conID1>
        [config.v2.json, file]
        [hostname, file]
        [..., file]
    ]
]   
]
\end{forest}

\caption{Docker Layered File System Structure Based on AUFS Storage Driver}

\floatfoot{White box stands for a file and blue box stands for a directory. 
The directory \textit{./aufs/mnt/<rootfs ID>/} is the mount points of the root file system for the containers. All the image layers are union mounted to this folder and provide a whole system view for the container.
The directory \textit{./aufs/diff/<layer ID>/} stores all the image layer contents from one specific layer identified by a local SHA256 ID, which we noted as <layer ID>. 
The file \textit{./aufs/layers/<rootfs ID>} is a file stores all the image layers' cached IDs for the container.
The 
%   ├── layers // Metadata of layers
%   │   ├── 1
%   │   ├── 2
%   │   └── 3
%   ├── diff  // Content of the layer
%   │   ├── 1  // Contains layers that need to be mounted for the id
%   │   ├── 2
%   │   └── 3
%   └── mnt    // Mount points for the rw layers to be mounted
%       ├── 1
%       ├── 2
%       └── 3
}
\label{fig:my_label2}
\end{figure*}
