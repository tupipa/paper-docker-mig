
https://docs.docker.com/engine/userguide/storagedriver/imagesandcontainers/#content-addressable-storage


Content addressable storage

Docker 1.10 introduced a new content addressable storage model. This is a completely new way to address image and layer data on disk. Previously, image and layer data was referenced and stored using a randomly generated UUID. In the new model this is replaced by a secure content hash.

The new model improves security, provides a built-in way to avoid ID collisions, and guarantees data integrity after pull, push, load, and save operations. It also enables better sharing of layers by allowing many images to freely share their layers even if they didn’t come from the same build.


Container and layers

The major difference between a container and an image is the top writable layer. All writes to the container that add new or modify existing data are stored in this writable layer. When the container is deleted the writable layer is also deleted. The underlying image remains unchanged.

Because each container has its own thin writable container layer, and all changes are stored in this container layer, this means that multiple containers can share access to the same underlying image and yet have their own data state. The diagram below shows multiple containers sharing the same Ubuntu 15.04 image.

The Docker storage driver is responsible for enabling and managing both the image layers and the writable container layer. How a storage driver accomplishes these can vary between drivers. Two key technologies behind Docker image and container management are stackable image layers and copy-on-write (CoW).



...........


Each of these layers is stored in its own directory inside the Docker host’s local storage area.

Versions of Docker prior to 1.10 stored each layer in a directory with the same name as the image layer ID. However, this is not the case for images pulled with Docker version 1.10 and later. For example, the command below shows an image being pulled from Docker Hub, followed by a directory listing on a host running version 1.9.1 of the Docker Engine.


$  docker pull ubuntu:15.04

15.04: Pulling from library/ubuntu
47984b517ca9: Pull complete
df6e891a3ea9: Pull complete
e65155041eed: Pull complete
c8be1ac8145a: Pull complete
Digest: sha256:5e279a9df07990286cce22e1b0f5b0490629ca6d187698746ae5e28e604a640e
Status: Downloaded newer image for ubuntu:15.04

$ ls /var/lib/docker/aufs/layers

47984b517ca9ca0312aced5c9698753ffa964c2015f2a5f18e5efa9848cf30e2
c8be1ac8145a6e59a55667f573883749ad66eaeef92b4df17e5ea1260e2d7356
df6e891a3ea9cdce2a388a2cf1b1711629557454fd120abd5be6d32329a0e0ac
e65155041eed7ec58dea78d90286048055ca75d41ea893c7246e794389ecf203

Notice how the four directories match up with the layer IDs of the downloaded image. Now compare this with the same operations performed on a host running version 1.10 of the Docker Engine.

$ docker pull ubuntu:15.04
15.04: Pulling from library/ubuntu
1ba8ac955b97: Pull complete
f157c4e5ede7: Pull complete
0b7e98f84c4c: Pull complete
a3ed95caeb02: Pull complete
Digest: sha256:5e279a9df07990286cce22e1b0f5b0490629ca6d187698746ae5e28e604a640e
Status: Downloaded newer image for ubuntu:15.04

$ ls /var/lib/docker/aufs/layers/
1d6674ff835b10f76e354806e16b950f91a191d3b471236609ab13a930275e24
5dbb0cbe0148cf447b9464a358c1587be586058d9a4c9ce079320265e2bb94e7
bef7199f2ed8e86fa4ada1309cfad3089e0542fec8894690529e4c04a7ca2d73
ebf814eccfe98f2704660ca1d844e4348db3b5ccc637eb905d4818fbfb00a06a

See how the four directories do not match up with the image layer IDs pulled in the previous step.

Despite the differences between image management before and after version 1.10, all versions of Docker still allow images to share layers. For example, If you pull an image that shares some of the same image layers as an image that has already been pulled, the Docker daemon recognizes this, and only pulls the layers it doesn’t already have stored locally. After the second pull, the two images will share any common image layers.

You can illustrate this now for yourself. Starting with the ubuntu:15.04 image that you just pulled, make a change to it, and build a new image based on the change. One way to do this is using a Dockerfile and the docker build command.





Run the docker ps command to verify the 5 containers are running.

 $ docker ps
 CONTAINER ID    IMAGE             COMMAND    CREATED              STATUS              PORTS    NAMES
 0ad25d06bdf6    changed-ubuntu    "bash"     About a minute ago   Up About a minute            stoic_ptolemy
 8eb24b3b2d24    changed-ubuntu    "bash"     About a minute ago   Up About a minute            pensive_bartik
 a651680bd6c2    changed-ubuntu    "bash"     2 minutes ago        Up 2 minutes                 hopeful_turing
 9280e777d109    changed-ubuntu    "bash"     2 minutes ago        Up 2 minutes                 backstabbing_mahavira
 75bab0d54f3c    changed-ubuntu    "bash"     2 minutes ago        Up 2 minutes                 boring_pasteur

The output above shows 5 running containers, all sharing the changed-ubuntu image. Each CONTAINER ID is derived from the UUID when creating each container.

List the contents of the local storage area.

 $ sudo ls /var/lib/docker/containers

 0ad25d06bdf6fca0dedc38301b2aff7478b3e1ce3d1acd676573bba57cb1cfef
 9280e777d109e2eb4b13ab211553516124a3d4d4280a0edfc7abf75c59024d47
 75bab0d54f3cf193cfdc3a86483466363f442fba30859f7dcd1b816b6ede82d4
 a651680bd6c2ef64902e154eeb8a064b85c9abf08ac46f922ad8dfc11bb5cd8a
 8eb24b3b2d246f225b24f2fca39625aaad71689c392a7b552b78baf264647373


