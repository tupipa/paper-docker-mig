
\subsection{Docker Layered Storage Management}\label{aufsIntroduction}

As we mentioned above, Docker uses layered storage for its containers. Each Docker image references a list of read-only layers that represent file system differences. Layers are stacked on top of each other to form a base for a container's root filesystem \cite{dockerlayer}. 
% \cite{https://docs.docker.com/engine/userguide/storagedriver/imagesandcontainers/#images-and-layers}

% \def\cnt{16}
% \def\cnt{3}
\def\boxh{0.5}
\def\boxw{6.5}
\begin{figure}
    \centering
    
\tikzset{
    block/.style = {rectangle, draw, text width=6.0cm, line width=0.75pt,minimum height=0.6cm},
    block_dash/.style = {block, dashed},
}
\begin{tikzpicture}[node distance=0cm, auto]
\node (st16) [block_dash,
anchor=north east, 
% anchor=south, 
label={[label distance=0.3cm]-180:R/W}]
{febfb1642ebeb25857bf2a9c558bf695-init};
\node (st15) [block ,below =0.1cm of st16,label={[label distance=0.3cm]-180:RO}] 
{fac86d61dfe33f821e8d0e7660473381};
\node (st14) [block ,below = of st15,label={[label distance=0.3cm]-180:RO}] 
{984034c1bb9c62ac63fff949a70d1c06};
\node (st13) [block ,below = of st14,label={[label distance=0.3cm]-180:RO}] 
{2de00a5b0fb59d8eb7301b7523d96d3e};
\node (st12) [block ,below = of st13,label={[label distance=0.3cm]-180:RO}] {0cff6d24b7f45835d42401ec28408b34};
\node (st11) [block ,below = of st12,label={[label distance=0.3cm]-180:RO}] {edc8651e3133e0c7ad4b143e7aa8a238};
\node (st10) [block ,below = of st11,label={[label distance=0.3cm]-180:RO}] {d3eb4d3be5e23cad808eaa419d1f5302};
\node (st9) [block ,below = of st10,label={[label distance=0.3cm]-180:RO}] {374b33d90d684194b17796430fdf8dee};
\node (st8) [block ,below = of st9,label={[label distance=0.3cm]-180:RO}] {76bbec466a0b0b306ab1374476dc11f4};
\node (st7) [block ,below = of st8,label={[label distance=0.3cm]-180:RO}] {1193d29ec78769d161661434acb6d4ed};
\node (st6) [block ,below = of st7,label={[label distance=0.3cm]-180:RO}] {788c79301df4cbac57ffb834d6abded8};
\node (st5) [block ,below = of st6,label={[label distance=0.3cm]-180:RO}] {1562b357dde64e8ed8202c61959d6b13};
\node (st4) [block ,below = of st5,label={[label distance=0.3cm]-180:RO}] {44e0c44aca426478341b9c9f87f05f34};
\node (st3) [block ,below = of st4,label={[label distance=0.3cm]-180:RO}] {27a0a4ac3045470182cf7c1f4c26e379};
\node (st2) [block,below = of st3,label={[label distance=0.3cm]-180:RO}] {87b1dd26596e8e78e294a47b6b3fc3e9};
\node (st1) [block ,below = of st2, label={ [label distance=0.3cm]-180:RO}] {80db20d8e37dc3795b17e0e59930a408
};
\end{tikzpicture}
    \caption{OpenFace Container's Image Layer Stack. Container's  ID is \textit{
    9ec79a095ef4db1fc5edc5e4059f5a10}. It's rootfs ID is stored it's meta data file \textit{state.json}. The stack list is stored in file \textit{./aufs/layers/febfb1642ebeb25857bf2a9c558bf695}. On the top is the container layer, the only writable storage layer fot this container.}
    \label{fig:stacklist}
\end{figure}













% \begin{figure}
%     \centering
% \begin{tikzpicture}
% \draw [dashed] (0,15*\boxh + 0.1) rectangle (\boxw,16*\boxh + 0.2) node[pos=.5,label={[label distance=0.3cm]-180:R/W}]
% {febfb1642ebeb25857bf2a9c558bf695-init};
% \draw (0,14*\boxh) rectangle (\boxw,15*\boxh) node[pos=.5,label={[label distance=0.3cm]-180:RO}] 
% {fac86d61dfe33f821e8d0e7660473381};
% \draw (0,13*\boxh) rectangle (\boxw,14*\boxh) node[pos=.5,label={[label distance=0.3cm]-180:RO}] 
% {984034c1bb9c62ac63fff949a70d1c06};
% \draw (0,12*\boxh) rectangle (\boxw,13*\boxh) node[pos=.5,label={[label distance=0.3cm]-180:RO}] 
% {2de00a5b0fb59d8eb7301b7523d96d3e};
% \draw (0,11*\boxh) rectangle (\boxw,12*\boxh) node[pos=.5,label={[label distance=0.3cm]-180:RO}] {0cff6d24b7f45835d42401ec28408b34};
% \draw (0,10*\boxh) rectangle (\boxw,11*\boxh) node[pos=.5,label={[label distance=0.3cm]-180:RO}] {edc8651e3133e0c7ad4b143e7aa8a238};
% \draw (0,9*\boxh) rectangle (\boxw,10*\boxh) node[pos=.5,label={[label distance=0.3cm]-180:RO}] {d3eb4d3be5e23cad808eaa419d1f5302};
% \draw (0,8*\boxh) rectangle (\boxw,9*\boxh) node[pos=.5,label={[label distance=0.3cm]-180:RO}] {374b33d90d684194b17796430fdf8dee};
% \draw (0,7*\boxh) rectangle (\boxw,8*\boxh) node[pos=.5,label={[label distance=0.3cm]-180:RO}] {76bbec466a0b0b306ab1374476dc11f4};
% \draw (0,6*\boxh) rectangle (\boxw,7*\boxh) node[pos=.5,label={[label distance=0.3cm]-180:RO}] {1193d29ec78769d161661434acb6d4ed};
% \draw (0,5*\boxh) rectangle (\boxw,6*\boxh) node[pos=.5,label={[label distance=0.3cm]-180:RO}] {788c79301df4cbac57ffb834d6abded8};
% \draw (0,4*\boxh) rectangle (\boxw,5*\boxh) node[pos=.5,label={[label distance=0.3cm]-180:RO}] {1562b357dde64e8ed8202c61959d6b13};
% \draw (0,3*\boxh) rectangle (\boxw,4*\boxh) node[pos=.5,label={[label distance=0.3cm]-180:RO}] {44e0c44aca426478341b9c9f87f05f34};
% \draw (0,2*\boxh) rectangle (\boxw,3*\boxh) node[pos=.5,label={[label distance=0.3cm]-180:RO}] {27a0a4ac3045470182cf7c1f4c26e379};
% \draw (0,\boxh) rectangle (\boxw,2*\boxh) node[pos=.5,label={[label distance=0.3cm]-180:RO}] {87b1dd26596e8e78e294a47b6b3fc3e9};
% \draw (0,0) rectangle (\boxw,\boxh) node[ pos=0.5,
% % anchor=east, 
% label={[label distance=0.3cm]-180:RO}] {80db20d8e37dc3795b17e0e59930a408
% };
% \end{tikzpicture}
%     \caption{Image Layer Stack Example}
%     \label{fig:stacklist}
% \end{figure}


% ~\\
% \bigskip
% \smallskip
\smallbreak

\subsubsection{\textit{Container Layer} and Base Image Layers}
~\smallbreak
When a new container is created, a new, thin, writable storage layer is created on top of the underlying read-only stack of image layers. The new layer on the top is called  \textit{container layer}. All changes made to the container -- such as creation, modification, or deletion of any file -- are written to this \textit{container layer}\cite{dockerlayer}.

For example, 
Figure~\ref{fig:stacklist} shows the stacked image layers for OpenFace application. The dashed box on the top is the container layer of OpenFace. All the underlying layers are \textit{base image layers}. To resolve the access request for a file name, the storage driver will search the file name from top layer towards the bottom layer, the first copy of the file will be returned for accessing, regardless the any other copies with same file name in the underlying layers.

% It greatly simplifies the process of setting environment for any software platform. 

% \subsection{Docker Container File System Structure and Synchronization based on AUFS storage driver} 


\smallbreak
\subsubsection{Image Layer ID Mapping}
~\smallbreak
Since Docker 1.10, all the image and the layers in it are addressed by secure content SHA256 hash IDs. 
This addressable image design is supposed to improve security by avoiding name collisions, and at the same time, to maintain data integrity after pull, push, load, and save operations. It also enables better sharing of layers by allowing many images to freely share their layers locally even if they didn't come from the same build~\cite{dockerlayer}. 

% However, due to the active development of Docker platform and its relatively young ages of history, there is no existing effort dived into how those addressable images worked by those SHA256 hash. 
% However, there is 
% original and cache IDs are used and maintained throughout the life cycle of one container, as well as the life cycle of the underlying docker daemon. 
%
% By , we explain how Docker platform addresses different storage layers and propose our methods to leverage those layers in the live migration of Docker containers.
% keeps track of those mapping relationship. 
%
Given that there is no existing effort dived into how those addressable images worked by those SHA256 hash, we have investigated into the source code of Docker and its storage drivers.
We find that there is an image layer ID mapping relationship which is not well documented yet: if the same image is downloaded from the same build on the cloud, Docker will mapping the layers' original layer IDs to a new secure content hash, called \textit{cache ID}. Every image layer's original ID will be replaced with its cache ID. From then on, Docker daemon will address the image layer by this cache ID when it creates, starts, stops, or checkpoints/restores a container. 
% shanhe: might need to explain, how this finding helps?

\smallbreak
\subsubsection{ID Matching Between Docker Host} \label{intro:idMatching}
~\smallbreak
The mapping of layer IDs exposes a challenge to address those layers when we migrate a container between different hosts. 
As we have found, the image layer on the cloud will have different cache IDs when downloaded to different Docker host. 
Therefore, if we want to share the common image layers between different Docker hosts, we need to resolve the image layers ID mapping problem according different Docker host. 

To address this problem, on each Docker host, we rebuild the mapping between its cache IDs to their original layer IDs by querying the Docker image database. 
% that have the same origin on two different Docker host.
Thus, we could match the image layers by its original IDs instead of the cache IDs exposed to the local Docker daemons. More details can be found in section \ref{design:idremapping}.

% In order to do this, we then need to generate a new image layer stack with different cache IDs but actually have the same content inside since they have the same original layer IDs. 


% In the Docker's root system, the /usr/lib/docker/ by default (or /usr/lib/docker/0.0/ for docker 1.10 version), 

\smallbreak
\subsubsection{Docker's Storage Driver Backend}
~\smallbreak

Docker delegates the task of managing the container's layered file system to its pluggable storage driver. 
The Docker storage driver is designed to provide a single unified view from a stack of image layers.
Users can choose from different kinds of storage drivers that is the best for their particular case of usage. 

%  https://github.com/docker/docker/blob/b248de7e332b6e67b08a8981f68060e6ae629ccf/daemon/graphdriver/aufs/aufs.go
% /*
% aufs driver directory structure
%   .
%   ├── layers // Metadata of layers
%   │   ├── 1
%   │   ├── 2
%   │   └── 3
%   ├── diff  // Content of the layer
%   │   ├── 1  // Contains layers that need to be mounted for the id
%   │   ├── 2
%   │   └── 3
%   └── mnt    // Mount points for the rw layers to be mounted
%       ├── 1
%       ├── 2
%       └── 3
% */

 \tikzset{/forest,
    % symlink/.append style={
    %   opacity=.25,
    %   text opacity=.5,
    %   before drawing tree={
    %     {tikz+={\draw [thick, -{>[]}] (!#1.west) ++(4pt,-1.5pt) arc (315:120:5pt);}}
    %   },
    % },
    % file/.append style={
    file/.style={
        % minimum width=0.5cm,
        % minimum height=0.5cm,
        % parent anchor=south,
        % child anchor=north,
        % sibling distance = 4cm,
        % dogeared,
        % fill=white!50,
        % draw,
        trapezium,
        trapezium angle=0,
        rounded corners=2pt,
        draw,
        fill=white!50,
    },
   }
  
  
\begin{figure*}
\centering
% http://tex.stackexchange.com/questions/5073/making-a-simple-directory-tree
\begin{forest}
for tree={
    minimum width=0.5cm,
    minimum height=0.5cm,
    parent anchor=south,
    child anchor=north,
    trapezium,
    trapezium angle=0,
    rounded corners=2pt,
    draw,
    fill=blue!50,
    % % http://tex.stackexchange.com/questions/278708/center-root-of-forest-tree
    % if level=0{ 
    % parent anchor=south,
    % child anchor=north,
    % align=center,
    % l=1cm,
    % fill=white,
    % minimum width=\linewidth,
    % inner xsep=0pt,
    % outer xsep=0pt,
    % }{},
    % myfolder,
}
% [, phantom, s sep = 1cm
[/var/lib/docker/0.0/
[containers
    [<conID>
        [config.v2.json, file]
        % [resolve.conf, file]
        [..., file]
    ]
    % [<conID1>
    %     % [config.v2.json, file]
    %     % % [hostname, file]
    %     % [..., file]
    % ]
]   
[aufs/
    [mnt/
        [<rootfs ID>/
            % [/boot/
            %     [...]
            % ]
            [/etc/
                [hostname, file]
                [hosts, file]
            ]
            [/home/
                % [username/
                %     [.bashrc, file]
                %  [...]
                % ]
            ]
            % [/.../]
        ]
    ]
    [layers/
        [<rootfs ID>, file]
    ]
    [diff/
        [<layer ID1>/
            [/etc/
                [hostname, file]
            ]
            % [/.../]
        ]
        [<layerID2>/
            [/etc/
                [hostname, file]
                [hosts, file]
            ]
        ]
    ]
]
[image/aufs/layerdb/sha256
    % [aufs
        % [layerdb
            % [mounts
            %     % [<conID>]
            % ]
            % [sha256
                [<O-layerID>
                    [cache-id, file]
                    % [diff,file]
                    [parent,file]
                    [..., file]
                ]
            % ]
        % ]
        % [imagedb]
    % ]
]
]
\end{forest}

\caption{Docker Layered File System Structure Based on AUFS Storage Driver}
\label{fig:aufs}
\end{figure*}


%  https://github.com/docker/docker/blob/b248de7e332b6e67b08a8981f68060e6ae629ccf/daemon/graphdriver/aufs/aufs.go
% /*
% aufs driver directory structure
%   .
%   ├── layers // Metadata of layers
%   │   ├── 1
%   │   ├── 2
%   │   └── 3
%   ├── diff  // Content of the layer
%   │   ├── 1  // Contains layers that need to be mounted for the id
%   │   ├── 2
%   │   └── 3
%   └── mnt    // Mount points for the rw layers to be mounted
%       ├── 1
%       ├── 2
%       └── 3
% */

\begin{figure}
\centering
% http://tex.stackexchange.com/questions/5073/making-a-simple-directory-tree
\begin{forest}
for tree={
    minimum width=0.5cm,
    minimum height=0.5cm,
    parent anchor=south,
    child anchor=north,
    trapezium,
    trapezium angle=0,
    rounded corners=2pt,
    draw,
    fill=blue!50,
    % % http://tex.stackexchange.com/questions/278708/center-root-of-forest-tree
    % if level=0{ 
    % parent anchor=south,
    % child anchor=north,
    % align=center,
    % l=1cm,
    % fill=white,
    % minimum width=\linewidth,
    % inner xsep=0pt,
    % outer xsep=0pt,
    % }{},
    % myfolder,
}
% [, phantom, s sep = 1cm
[/var/run/docker/execdriver/native/
    [<conID1>
        [state.json, file]
    ]
    [<conID2>
        [state.json, file]
    ]
]
% ]
\end{forest}

\caption[Caption for LOF]{Runtime Data for Containers\protect\footnotemark 
% -dev, the latest Docker (version 17.04.0-ce, build 4845c56) has this directory changed to \textit{/var/run/docker/libcontainerd/containerd}}
% \footnote{This is for Docker 1.10\-dev, the latest Docker (version 17.04.0\-ce, build 4845c56) has this directory changed to \textit{/var/run/docker/libcontainerd/containerd}}
}
\label{fig:aufs-runtime}
\end{figure}

% \footnotetext{This is for Docker 1.10-dev, the latest Docker (version 17.04.0-ce, build 4845c56) has this directory changed to \textit{/var/run/docker/libcontainerd/containerd}}

In order to get more details of how those addressable images work, we investigate the source code of Docker system along with one of its most popular storage driver, AUFS. 
% We will show our findings inside the structure of the Docker's storage systems in this section.
For other storage drivers like Btrfs, Device Mapper, overlay, and ZFS, they implement the management of image layers and the \textit{container layer} in their own ways, but our framework could also be extended to those drivers. Due to limited time and space, we only conduct experiments on AUFS. The following section will discuss the inner details about our findings inside the Docker's AUFS storage driver.

% TODO: more details in images management, including the mapping details. Draw a docker root directory structure Figure

% \paragraph{AUFS storage}\mbox{}\\

\subsection{AUFS Storage: A Case Study}
The default storage driver on Docker is Advanced multi-layered Unification FileSystem (AUFS). Therefore, we take this as an example to introduce the layered images management.

AUFS storage driver implements Docker image layers by a union mount system. Union mount is a way of combining numerous directories into one directory that looks like it contains the contents from all the them \cite{aufs}. Docker uses union filesystem to merge all image layers together and presents them as one single read-only view at a union mount point. If there are duplicate identities (i.e. file names) in different layers, only the one on the highest layer is accessible.

Figure~\ref{fig:aufs} shows the Docker storage structure based on the AUFS driver. White box stands for a file and blue box stands for a directory. Since all directories share the same parent path \textit{/var/lib/docker/0.0/}, we will use `.' to represent this common directory in the following discussion. 

AUFS driver use three main directories to manage  image layers: \textit{layers/}, \textit{diff/}, and \textit{mnt/}:  directory \textit{layers/} contains the metadata of how image layers are stacked together; directory \textit{diff/} stores the content data for each layers; directory \textit{mnt/} contains the mount point of the root file system for each the container. 


\smallbreak
\subsubsection{Container's Image Layer Stack List}
~\smallbreak
% \smallbreak  \subsubsection{Container's Image Layer Stack List:}


We know that each Docker image contains several image layers. Those image layers are addressed by their SHA256 content hash IDs. Each Docker image has a list of layer IDs in the order of how they stacked from top to bottom.
There are two files, \textit{./aufs/layers/<rootfs ID>-init} and \textit{./aufs/layers/<rootfs ID>}, both of which store a list of layer IDs. The former stores IDs of all initial image layers when the container is created. The latter stores IDs of the newly created layers in addition to the initial layers.  In Figure \ref{fig:aufs}, we use \textit{./aufs/layers/<rootfs ID>(-init)} as the notion of the two files. 

% The file of \textit{./aufs/layers/<rootfs ID>-init} stores a list of SHA256 IDs of all image layers that will be union mounted as the container's root file system. 
For example, for the container \textit{OpenFace} with rootfs ID of 
\textit{
febfb16-42ebeb25857bf2a9c558bf695
\footnotetext{This is for Docker 1.10-dev, the latest Docker (version 17.04.0-ce, build 4845c56) has the runtime data directory changed to \textit{/var/run/docker/libcontainerd/containerd/}.}
\footnote{SHA256 ID has 64 hexadecimal characters, here we truncate it to 32 hexadecimal characters in order to save space.}
}, it's initial layer stack is stored in the file \textit{./aufs/ layers/febfb1642ebeb25857bf2a9c558bf695-init}. It  contains all layers in the downloaded container image \textit{bamos/openface}\footnote{https://hub.docker.com/r/bamos/openface/.}. These layers will be read-only throughtout the whole life cycle of the container. Once a new layer is created, i.e. the \textit{container layer}, the layer ID will be listed on the top line of the file \textit{./aufs/ layers/febfb1642ebeb25857bf2a9c558bf695}.

% The other file \textit{./aufs/layers/<rootfs ID>} which stores this list of image layer IDs for that Docker image as well as the ID of the writable \textit{container layer}.  The container layer's ID is used to address the thin writable layer for that container. This \textit{container layer ID} is the same as the \textit{<rootfs ID>}. 

When the Docker daemon starts or restores a container, it will refer to those two files to get a list of all underlying Docker image layer IDs and the container layer ID. Then it will resolve those addressable IDs and union mount all those layer stacks together in the specific order. After this, the container will get the full view of its root file system under the root mount point. 
We find that this file behaves like an important handler for all the union file systems for the container. If this file is missing, one container will not be able to union mount the root file system correctly.

% --------------

\smallbreak 
\subsubsection{Image Layer Content Directory}
~\smallbreak
AUFS manages the content of image layers in the directory of \textit{./aufs/diff/}. The directory
 \textit{./aufs/diff/<layer ID>/} stores all the files inside the specific layer identified by its \textit{<layer ID>}. This can be either a readonly image layers or layers newly created for a container. If \textit{<layer ID>} is the same as \textit{<rootfs ID>} of one container, then this directory is where the content of \textit{container layer} stores, i.e. all the file system changes of the container will be stored under this directory. 

\smallbreak 
\subsubsection{Unified Mount Point} 
~\smallbreak
The directory \textit{./aufs/mnt/<rootfs ID>/}  is the mount point of the container's root file system. All image layers are union mounted to this folder and provide a single file system view for the container. 
For example, as shown in Figure \ref{fig:aufs}, when a container is created based on a Linux image, its mount point will contain the root directory contents like \textit{/usr/, /home/, /boot/, etc. }.
All those directories are mounted from its underlying layered storage, including both the read only image layers downloaded from the registry and the newly created layers for the container. 
Since this directory is a mount point for a running container's file system, it will be only available when the container is running. If the container stops running, all the image layers will be unmounted from this mount point. So it will become an empty directory.

Here, the name of the root file system directory, \textit{<rootfs ID>}, is the same as the name of the \textit{container layer} for this container.

\smallbreak  
\subsubsection{Layer ID Mapping}  \label{intro:aufs:layerIDMapping}
~\smallbreak
Until now, the layer IDs we have discussed above are just local SHA256 IDs, or the so called cache IDs, which are generated dynamically when each image layer is downloaded by \textit{`docker pull'} command. From then on, Docker daemon will address the image layer use the cache ID instead of its original layer ID (noted as O-layerID in this paper).

We find the Docker storage system maintains a mapping relationship between the original layer IDs and its cache IDs. All the cached IDs of image layers are stored in the \textit{/image/aufs/layerdb/sha256} directory.
For example, the file \textit{./image/aufs/layerdb/sha256/<O-layerID>/cache-id} shown in Figure~\ref{fig:aufs} stores the cache ID of the image with original ID <O-layerID>. For example, if a hash ID \textit{
fac86d61dfe33f821e8d0e7660473381} is stored in the file of \textit{./image/ aufs/ layerdb/ sha256/ 6384c447ddd6cd859f9be3b53f8b015c/cache-id}, this means there is an image layer with an original ID of \textit{
6384c-447ddd6cd859f9be3b53f8b015c} and it's cache ID is \textit{
fac86d61dfe33f-821e8d0e7660473381}.


% However, due to the active development of Docker community and its relatively young ages of history, there is no available articles illustrating how those original and cache IDs are used and maintained throughout the life cycle of one container, as well as the life cycle of the underlying docker daemon. By investigating into the source code of Docker as well as its storage system, we find out how Docker platform keeps track of those mapping relationship. 

\smallbreak  
\subsubsection{Container Configuration and Runtime State}
~\smallbreak

There are several directories that store the configuration files and runtime data. Figure~\ref{fig:aufs-runtime} shows the runtime data directories for each containers. For one container with ID of \textit{<conID>}, there will be a JavaScript object notation (JSON)  file \textit{state.json} that stores the runtime state of the container. For example, the \textit{init pid} of the containers' processes is identified by key ``\textit{init\_process\_pid}'', and the root file system mount point path can be found via key ``\textit{rootfs}''. There are also some runtime cgroup and namespace meta data, etc.. 

Along with the runtime data directory, there is another directory named \textit{/var/lib/docker/0.0/containers/<conID>} that contains the configuration files for each container. The directory is shown in Figure~\ref{fig:aufs}. We use \textit{<conID>} as a notation of the container's hash ID. For example, from the file of \textit{ config.v2.json}, we can find the container's creation time, the command that was run once the container was created, etc..


% TODO: more details in AUFS. Draw an aufs driver directory structure Figure

%   ├── layers // Metadata of layers
%   │   ├── 1
%   │   ├── 2
%   │   └── 3
%   ├── diff  // Content of the layer
%   │   ├── 1  // Contains layers that need to be mounted for the id
%   │   ├── 2
%   │   └── 3
%   └── mnt    // Mount points for the rw layers to be mounted
%       ├── 1
%       ├── 2
%       └── 3
