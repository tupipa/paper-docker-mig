\usepackage{booktabs} % For formal tables


% \usepackage{cite}
\usepackage{pbox}
\usepackage{array}

\newcolumntype{P}[1]{>{\centering\arraybackslash}p{#1}}
\newcolumntype{M}[1]{>{\centering\arraybackslash}m{#1}}
\usepackage{amsmath}
\usepackage{algorithmic}
% \usepackage{array}
\usepackage{graphicx}
% \usepackage{subfigure}
% \usepackage{subcaption}
\usepackage{tikz}
\usetikzlibrary{positioning, shapes, shapes.geometric, arrows, chains, calc,intersections, patterns}

\usepackage{multirow}

% https://tex.stackexchange.com/questions/50934/how-to-mark-a-node-in-a-graph-with-a-pattern
% \pgfdeclarepatternformonly{soft horizontal lines}{\pgfpointorigin}{\pgfqpoint{100pt}{1pt}}{\pgfqpoint{100pt}{3pt}}%
% {
%   \pgfsetstrokeopacity{0.3}
%   \pgfsetlinewidth{0.1pt}
%   \pgfpathmoveto{\pgfqpoint{0pt}{0.5pt}}
%   \pgfpathlineto{\pgfqpoint{100pt}{0.5pt}}
%   \pgfusepath{stroke}
% }

% environ: scale tikzpicture to textwidth
%   https://tex.stackexchange.com/questions/6388/how-to-scale-a-tikzpicture-to-textwidth
\usepackage{environ}

% \usepackage{caption}
\usepackage{pgfplots}
\pgfplotsset{compat=1.12}
\usepackage{xr}
%\usepackage{mathtools}
\usepackage{amsmath}

\usepackage{url}
\usepackage[utf8]{inputenc} 
% \usepackage{fourier} 

% \usepackage{enumitem}
\usepackage[inline, shortlabels]{enumitem}
% \hyphenation{op-tical net-works semi-conduc-tor}

% http://tex.stackexchange.com/questions/56529/note-below-figure
% \usepackage[capposition=top]{floatrow}

% \usepackage{forest}
 \usepackage[edges]{forest}

% \usepackage[T1]{fontenc}

\usepackage{caption}
\usepackage{subcaption}